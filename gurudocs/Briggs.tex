%	$Id: ternary.tex 14379 2015-06-23 02:37:32Z pwessel $
% Internal documentation for SOliving Briggs coefficients
% Paul Wessel, April 21, 2015.
\documentclass[12pt,letterpaper,margin=0.5in]{article}
\usepackage{times}
\usepackage{graphicx}
\usepackage{breqn}
\usepackage[margin=1in]{geometry}
\begin{document}

\section{BRIGGS COEFFICIENT SOLUTIONS}

Testing reveals that our implementation of the off-node data constraints is incorrect
for some quadrants.  Our implementation holds for the first quadrant but then we
use this geometry for all quadrants by only using the absolute value of the point's
coordinates relative to the central node.  This seems incorrect.  Here are the
solutions obtained by symbolic Matlab [See Briggs.m script]:
\subsection{First quadrant}
Here the data constraint point E has relative coordinates ($u_E, v_E$).  The other four points used are
nodes on the grid and thus have relative $x$- and $y$-coordinates belonging to the set \{$-1, 0, +1$\}, depending on
which node we consider. The four nodes used in the
construction for quadrant 1 are numbered \{1, 5, 9, 10\} in the code and relative to the central (0,0) node they
are identified as \{NW, W1, S1, SE\}. These form the first four columns in the matrix, while point E
forms the 5th column (this format is kept for all quadrants).  Hence, the matrix equation
for $b_k$ becomes
\begin{equation}
\left[ {\begin{array}{*{20}{r}}
{ - 1}&{ - 1}&0&1&{{u_E}}\\
1&0&{ - 1}&{ - 1}&{{v_E}}\\
1&1&0&1&{u_E^2}\\
{ - 1}&0&0&{ - 1}&{{u_E}{v_E}}\\
1&0&1&1&{v_E^2}
\end{array}} \right] \cdot \left[ {\begin{array}{*{20}{c}}
{{b_1}}\\
{{b_2}}\\
{{b_3}}\\
{{b_4}}\\
{{b_5}}
\end{array}} \right] = \left[ {\begin{array}{*{20}{c}}
0\\
0\\
2\\
0\\
2
\end{array}} \right].
\end{equation}
This is identical to our published case.  Introducing $\Delta = \left (u_E + v_E + 1 \right)\left (u_E + v_E\right)$ we find
the solution
\begin{equation}
\left[ {\begin{array}{*{20}{c}}
{{b_1}}\\
{{b_2}}\\
{{b_3}}\\
{{b_4}}\\
{{b_5}}
\end{array}} \right] = \frac{1}{\Delta }\left[ {\begin{array}{*{20}{c}}
{u_E^2 - v_E^2 + 2{u_E}{v_E} + {u_E} - {v_E}}\\[4pt]
{2\left( {{v_E} - {u_E} + 1} \right)\left( {{u_E} + {v_E}} \right)}\\[4pt]
{2\left( {{u_E} - {v_E} + 1} \right)\left( {{u_E} + {v_E}} \right)}\\[4pt]
{v_E^2 - u_E^2 + 2{u_E}{v_E} - {u_E} + {v_E}}\\[4pt]
4
\end{array}} \right]
\label{eq:upper}
\end{equation}
\subsection{Second quadrant}
Here the point E has relative coordinates ($-u_E, v_E$).  The four nodes used in the
construction for quadrant 2 are numbered \{3, 6, 9, 8\} in the code and relative to the central (0,0) node they
are identified as \{NE, E1, S1, SW\}. The matrix equation for $b_k$ becomes
\begin{equation}
\left[ {\begin{array}{*{20}{r}}
1&1&0&{ - 1}&{ - {u_E}}\\
1&0&{ - 1}&{ - 1}&{{v_E}}\\
1&1&0&1&{u_E^2}\\
1&0&0&1&{ - {u_E}{v_E}}\\
1&0&1&1&{v_E^2}
\end{array}} \right] \cdot \left[ {\begin{array}{*{20}{c}}
{{b_1}}\\
{{b_2}}\\
{{b_3}}\\
{{b_4}}\\
{{b_5}}
\end{array}} \right] = \left[ {\begin{array}{*{20}{c}}
0\\
0\\
2\\
0\\
2
\end{array}} \right].
\end{equation}
The solution is identical to that for quadrant 1.
\subsection{Third quadrant}
Here the point E has relative coordinates ($-u_E, -v_E$).  The four nodes used in the
construction for quadrant 3 are numbered \{1, 2, 6, 10\} in the code and relative to the central (0,0) node they
are identified as \{NW, N1, E1, SE\}. The matrix equation for $b_k$ becomes
\begin{equation}
\left[ {\begin{array}{*{20}{r}}
{ - 1}&0&1&1&{ - {u_E}}\\
1&1&0&{ - 1}&{ - {v_E}}\\
1&0&1&1&{u_E^2}\\
{ - 1}&0&0&{ - 1}&{{u_E}{v_E}}\\
1&1&0&1&{v_E^2}
\end{array}} \right] \cdot \left[ {\begin{array}{*{20}{c}}
{{b_1}}\\
{{b_2}}\\
{{b_3}}\\
{{b_4}}\\
{{b_5}}
\end{array}} \right] = \left[ {\begin{array}{*{20}{c}}
0\\
0\\
2\\
0\\
2
\end{array}} \right],
\end{equation}
with solution
\begin{equation}
\left[ {\begin{array}{*{20}{c}}
{{b_1}}\\
{{b_2}}\\
{{b_3}}\\
{{b_4}}\\
{{b_5}}
\end{array}} \right] = \frac{1}{\Delta }\left[ {\begin{array}{*{20}{c}}
{v_E^2 - u_E^2 + 2{u_E}{v_E} + {v_E} - {u_E}}\\[4pt]
{2\left( {{u_E} - {v_E} + 1} \right)\left( {{u_E} + {v_E}} \right)}\\[4pt]
{2\left( {{v_E} - {u_E} + 1} \right)\left( {{u_E} + {v_E}} \right)}\\[4pt]
{u_E^2 - v_E^2 + 2{u_E}{v_E} - {v_E} + {u_E}}\\[4pt]
4
\end{array}} \right].
\label{eq:lower}
\end{equation}
Apart from $b_5$, this solution is clearly different from the first two quadrants and herein lies our problem.
\subsection{Fourth quadrant}
Here the point E has relative coordinates ($-u_E, -v_E$).  The four nodes used in the
construction for quadrant 4 are numbered \{3, 2, 5, 8\} in the code and relative to the central (0,0) node they
are identified as \{NE, N1, W1, SW\}. The matrix equation for $b_k$ becomes
\begin{equation}
\left[ {\begin{array}{*{20}{r}}
1&0&{ - 1}&{ - 1}&{{u_E}}\\
1&1&0&{ - 1}&{ - {v_E}}\\
1&0&1&1&{u_E^2}\\
1&0&0&1&{ - {u_E}{v_E}}\\
1&1&0&1&{v_E^2}
\end{array}} \right] \cdot \left[ {\begin{array}{*{20}{c}}
{{b_1}}\\
{{b_2}}\\
{{b_3}}\\
{{b_4}}\\
{{b_5}}
\end{array}} \right] = \left[ {\begin{array}{*{20}{c}}
0\\
0\\
2\\
0\\
2
\end{array}} \right].
\end{equation}
The solution is identical to that for quadrant 3.
Comparisons of the solution for the upper and lower halves reveal the differences are all proportional to
($u_e - v_E$), hence there are no differences when data constraints lie along the 45-degree diagonals.
Inspection of equations (\ref{eq:upper}) and (\ref{eq:lower}) shows that they have the same form
but $u_E$ and $v_E$ have been interchanged.  We can therefore simply use (\ref{eq:upper}) for all cases
as long as we swap $u_E$ and $v_E$ when our point falls in quadrants 3--4.
\end{document}
