%	$Id$
% Internal documentation for Grid pad boundary conditions
% Paul Wessel, May 24, 2023.
\documentclass[12pt,letterpaper,margin=0.5in]{article}
\usepackage{times}
\usepackage{epsfig}
\usepackage{amsmath}
\usepackage{graphicx}
\usepackage{breqn}
\usepackage[margin=1in]{geometry}
\newcommand{\PDFfig}[4][tbp]{\begin{figure}[#1] \centering \epsfig{figure=#2,width=#4} \caption{{\small #3}} \label{fig:#2} \end{figure}}
\pagenumbering{arabic}

\begin{document}
\section{SETTING PAD NODES WHEN READING SUBSETS}

The problem we discuss is the mixed data and natural BC on just one side of a grid.  The grid
has dimensions $n_y \times n_x$.

\subsection{Setting the BCs in the west pad}
\label{sec:west}
\PDFfig[h]{nodes}{Memory layout (nodes and pad) when reading a subset from a grid file.  The desired region (fat lines)
has data nodes ($d_{i,j}$) and because the original grid file extends further in the south, east, and north
directions the pad also has actual data nodes serving as boundary conditions. All nodes left of the vertical thick
line have no data and are initially zeros.  BC nodes to be discussed in the text are colored. Gray box indicates the 
4x4 set of nodes involved in the BCR convolution to estimate a value at the plus symbol.}{4.5in}

\subsection{Setting the BCs in the west pad one step out}
Per the {\bf surface} notes, we know we have two boundary conditions we can satisfy on the west in the case of
natural conditions. The first is the requirement that the Laplacian curvature at the west edge is zero, i.e.,
\begin{equation}
	\nabla^2 z = 0.
\end{equation}
This results in a finite difference expression that relates inside, edge and outside nodes, which we can then
rearrange to solve for the outside ($z_{-1,0}$) node.  Here, subscripts $_{0,0}$ indicates a particular $d$ point on the
western edge, and the other subscripts are relative to this anchor point:
\begin{equation}
	z_{-1,0} = 4 d_{0,0} - d_{1,0} - d_{0,1} - d_{0,-1}.
	\label{eq:BC1}
\end{equation}
As this equation is applied for all rows $j = 0$ up to and including $j = n_y - 1$, we fill that column with values thus:
\begin{equation}
	z_{-1,j} = 4 d_{0,j} - d_{1,j} - d_{0,j+1} - d_{0,j-1}.
	\label{eq:BC1j}
\end{equation}

\subsection{Setting the BCs in the corners}
At the corner of the grid we require 
\begin{equation}
	\frac{\partial^2}{\partial x \partial y} z = 0.
\end{equation}
This yields the finite difference expression for the NW green node:
\begin{equation}
	z_{-1,-1} = d_{1,-1} - d_{1,1} - z_{-1,1}
	\label{eq:BCc}
\end{equation}
The corner nodes must be set via (\ref{eq:BCc}) before we can move to the $z_{-2,j}$ nodes.

\subsection{Setting the BCs in the west pad two steps out}
The third boundary condition says the side-normal (here positive $x$-direction) derivative of the Laplacian should also be
zero, i.e.,
\begin{equation}
	\frac{\partial}{\partial x} \left (\nabla^2 z \right ) = 0.
\end{equation}
This expression can also be written as finite differences and we may solve for the outside node:
\begin{equation}
	z_{-2,0} = d_{1,0} + 5(z_{-1,0} - d_{0,0}) + (d_{0,-1} - z_{-1,-1}) + (d_{0,1} - z_{-1,1})
	\label{eq:BC2}
\end{equation}
This equation is then applied to all the rows with BC2 labels we fill that column with values.
Obviously, (\ref{eq:BC1}) needs to be applied for all the $z_{-1,j}$ for all rows $j = 0$ up to
and including $j = n_y - 1$ since (\ref{eq:BC2}) references nodes that must be set first, i.e.,
\begin{equation}
	z_{-2,j} = d_{1,j} + 5(z_{-1,j} - d_{0,j}) + (d_{0,j-1} - z_{-1,j-1}) + (d_{0,j+1} - z_{-1,j+1}).
	\label{eq:BC2}
\end{equation}

\subsection{Setting the red- and yellow-colored nodes}
The red and yellow nodes are never listed in any of our BCs (except when filling data) so to prevent them
from being included in the BCR convolution we must set
\begin{equation}
	z_{-2,2} = z_{-1,2} = z_{-2,-1} = \mbox{NaN}.
	\label{eq:BC2}
\end{equation}

\subsection{The order of BCs}
In conclusion, we first apply (\ref{eq:BC1j}), then the corner (\ref{eq:BCc}), and finally (\ref{eq:BC2}).
Applying the corner BC first (GMT 6.4.0 and earlier) means some nodes are still zero for this scenario.

\end{document}.
