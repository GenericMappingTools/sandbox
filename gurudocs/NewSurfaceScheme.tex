%	$Id: ternary.tex 14379 2015-06-23 02:37:32Z pwessel $
% Internal documentation for SOliving Briggs coefficients
% Paul Wessel, April 21, 2015.
%\documentclass[12pt,letterpaper,margin=0.5in]{report}
\documentclass[12pt,letterpaper]{article}
\usepackage{times}
\usepackage{graphicx}
\usepackage{amsmath}
%\usepackage[margin=0.5in]{geometry}
%\usepackage{lscape}
%\textheight = 9 in
%\topmargin = -1 in
\begin{document}

\section{Determine $C_{00}$ using interpolating biquadratic}
Briggs used a 2-D Taylor expansion from the center node to an off-node data constraint
to include such data constraints in a minimum curvature spline.
As an alternative approach to incorporating such off-node data constraints in surface,
consider fitting the biquadratic model
\begin{equation}
	z(x,y) = c_1 + c_2 x + c_3 y + c_4 xy + c_5 x^2 + c_6 y^2 + c_7 x^2y + c_8 y^2x + c_9 x^2y^2
	\label{eq:biquadratic}
\end{equation}
to the nearest 8 nodes plus a single data constraint in the vicinity of ($0,0$).  Using the nodes numbered 1, 2, 3, 5, 6, 8, 9, 10
(i.e., NW, N1, NE, W1, E1, SW, S1, SE relative to the center node) we obtain the following matrix equation for the 9 unknown coefficients $c_k$ of the exact interpolation:
\begin{equation}
\left[ {\begin{array}{*{20}{c}}
1&{ - 1}&1&{ - 1}&1&1&1&{ - 1}&1\\
1&0&1&0&0&1&0&0&0\\
1&1&1&1&1&1&1&1&1\\
1&{ - 1}&0&0&1&0&0&0&0\\
1&1&0&0&1&0&0&0&0\\
1&{ - 1}&{ - 1}&1&1&1&{ - 1}&{ - 1}&1\\
1&0&{ - 1}&0&0&1&0&0&0\\
1&1&{ - 1}&{ - 1}&1&1&{ - 1}&1&1\\
1&x&y&{xy}&{{x^2}}&{{y^2}}&{{x^2}y}&{{y^2}x}&{{x^2}{y^2}}
\end{array}} \right] \cdot \left[ {\begin{array}{*{20}{c}}
{{c_1}}\\
{{c_2}}\\
{{c_3}}\\
{{c_4}}\\
{{c_5}}\\
{{c_6}}\\
{{c_7}}\\
{{c_8}}\\
{{c_9}}
\end{array}} \right] = \left[ {\begin{array}{*{20}{c}}
{{z_1}}\\
{{z_2}}\\
{{z_3}}\\
{{z_4}}\\
{{z_5}}\\
{{z_6}}\\
{{z_7}}\\
{{z_8}}\\
z
\end{array}} \right].
\end{equation}
Here, $z_1$ -- $z_8$ are the node values for the selected 8 grid nodes above; these all have $x$ and $y$ coordinates
in the \{$-1, 0, +1$\} set.  The single off-node data constraint is given by ($x,y,u$).
We only need (\ref{eq:biquadratic}) to evaluate $C_{00} = \nabla^2 z \rvert _{0, 0} = c_5 + c_6$ as all other terms will be zero.
This expression relates the curvature of the surface at the center node to our off-node data point.  Letting MATLAB do the algebra,
we obtain
\begin{equation}
C_{00} = \frac{D}{2(x^2-1)(y^2-1)}
\label{eq:biquad}
\end{equation}
with
\begin{equation}
\begin{split}
D = z_2 + z_4 + z_5 + z_7 - 4u - 2x z_4 + 2 x z_5 + 2 y z_2 - 2 y z_7 - x^2 z_2   \\
    + x^2 z_4 + x^2 z_5 - x^2 z_7 + y^2 z_2 - y^2 z_4 - y^2 z_5 + y^2 z_7 - x y^2 z_1    \\
    + x^2 y z_1 - 2 x^2 y z_2 + x y^2 z_3 + x^2 y z_3 + 2 x y^2 z_4 - 2 x y^2 z_5 - x y^2 z_6  \\
    - x^2 y z_6 + 2 x^2 y z_7 + x y^2 z_8 - x^2 y z_8 + x^2 y^2 z_1 - x^2 y^2 z_2 + x^2 y^2 z_3   \\
    - x^2 y^2 z_4 - x^2 y^2 z_5+ x^2 y^2 z_6 - x^2 y^2 z_7 + x^2 y^2 z_8 - x y z_1 + x y z_3 \\
    + x y z_6 - x y z_8.
\end{split}
\end{equation}
which we can reorganize to
\begin{equation}
\begin{split}
D = (z_2 + z_4 + z_5 + z_7 - 4u)   \\
	+ 2x (z_5 - z_4) + 2y (z_2 - z_7) + x y (z_3 + z_6 - z_1 - z_8) \\
	- x^2 (z_4 + z_5 - z_2 - z_7) + y^2 (z_2 - z_4 - z_5 + z_7)    \\
	+ x^2 y (z_1 - 2 z_2 + z_3 - z_6 + 2 z_7 - z_8) \\
	- x y^2 (z_1 + z_3 + 2 z_4 - 2 z_5 - z_6 + z_8)  \\
	+ x^2 y^2 (z_1 - z_2 + z_3 - z_4 - z_5 + z_6 - z_7 + z_8).
\end{split}
\end{equation}
From our Geophysics [1990] paper we know the desired surface must satisfy
\begin{equation}
\left( 1 - T_I\right)\nabla^4 z - T_I \nabla^2 z = \left( 1 - T_I\right)\nabla^2 C - T_I \nabla^2 z = 0.
\label{eq:biharm}
\end{equation}
We also derived
\begin{equation}
\nabla^2 C \rvert _{0, 0} = C_{10} + C_{-10} + \alpha^2(C_{01} + C_{0-1}) - 2 (1+\alpha^2)C_{00}.
\label{eq:curv}
\end{equation}
and
\begin{equation}
C_{00} = \nabla^2 z \rvert _{0, 0} = z_{10} + z_{-10} + \alpha^2(z_{01} + z_{0-1}) - 2 (1+\alpha^2)z_{00}.
\end{equation}
We will insert (\ref{eq:curv}) into (\ref{eq:biharm}) and replace $C_{00}$ with the value obtained from (\ref{eq:biquad}), then algebraically solve for $z_{00}$.
This gives
\begin{equation}
\begin{split}
z_{00} = \frac{1}{2(1+\alpha^4)} \{ 2(1+\alpha^2) \left [ z_{10}+z_{-10}+\alpha^2(z_{01}+z_{0-1})\right ] \\
	-2\alpha^2\left [z_{11} + z_{1-1} + z_{-11} + z_{-1-1}\right ] - \left [z_{20} + z_{-20} + \alpha^4 (z_{02} + z_{0-2})\right ] \\
	+ \left (\frac{T_I}{1-T_I} + 2(1+\alpha^2)\right )C_{00} \}.
\label{eq:offnode}
\end{split}
\end{equation}
Thus, we will use (\ref{eq:offnode}) instead of (A-7) when updating nodes with nearby data constraints.
\end{document}
