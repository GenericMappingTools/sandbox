%	$Id$
% Internal documentation for SOliving Briggs coefficients
% Paul Wessel, April 21, 2015.
\documentclass[12pt,letterpaper,margin=0.5in]{article}
\usepackage{times}
\usepackage{epsfig}
\usepackage{amsmath}
\usepackage{graphicx}
\usepackage{breqn}
\usepackage[margin=1in]{geometry}
\newcommand{\PDFfig}[4][tbp]{\begin{figure}[#1] \centering \epsfig{figure=#2,width=#4} \caption{{\small #3}} \label{fig:#2} \end{figure}}
\pagenumbering{arabic}

\begin{document}
\section{THE SURFACE ALGORITHM}

Smith and Wessel [1990] (here SW90) presented an
algorithm for gridding with continuous curvature splines in tension.
Here, we discuss technical details of their implementation and
correct errors in SW90. Introduced in GMT 1,
it used a Fortran west-to-east column arrangement with a row index increasing northward.
Since GMT 2, we have been using a scanline arrangements with rows arranged from north
to south with a column index increasing eastward.  We would like to recast
the algorithm to use this node arrangement to avoid a matrix transpose
and to take advantage of GMT library functions that
expect scanline orientated grids.  Sections below discusses various aspects of the
algorithm and proposed improvements.

\section{BRIGGS COEFFICIENT SOLUTIONS}

\PDFfig[h]{off_surface_nodes}{The four quadrants (1-4) where a data constraint (E) may be located.  The Briggs
solution published in SW90 is defined for the first quadrant (pink) only.  Other quadrants must be rotated
so that point E falls into the first quadrant and A--D can be identified.  Black square is the central node (0,0).}{5in}

Most data constraints are not likely to fall exactly (or even very close) to the
desired output nodes so we must find a way to include them in the finite difference
solution.  Figure~\ref{fig:off_surface_nodes} shows the four cases we must consider since
a data constraint may lie in any of the four quadrants surrounding the central node.
We follow Briggs [1974] in connecting the off-node point via a
second-order Taylor expansion from the central node ($z_{00}$) to a nearby point $z_k$:
\begin{equation}
	z_k = z_{00} + \xi_k \frac{\partial z}{\partial x} + \eta_k \frac{\partial z}{\partial x} \
	+ \frac{1}{2} \xi^2_k \frac{\partial^2 z}{\partial x^2} + \xi_k \eta_k \frac{\partial^2 z}{\partial x \partial y} \
	+ \frac{1}{2}  \eta^2_k \frac{\partial^2 z}{\partial y^2}.
\end{equation}
Here, the coordinates of the off-node points ($\xi_k, \eta_k$) are defined on a normalized
grid where the original grid cell dimension $\Delta x$ by $\Delta y$ is mapped to a unit square, i.e.,
\begin{equation}
\xi_k = \frac{x_k-x_{00}}{\Delta x}, \quad \alpha \eta_k = \frac{y_k-y_{00}}{\Delta y} \mbox{ or } \eta_k = \frac{y_k-y_{00}}{\Delta x},
\end{equation}
where $\alpha = \Delta x/ \Delta y$ is the grid spacing anisotropy (i.e., the inverse of what was stated in SW90).
The Briggs scheme is to evaluate this expansion at the five distinct points A--E, here represented as $(\xi_k, \eta_k), k = 1,5$.
We then multiply each expansion by a real number $b_k$ and sum the five expressions:
\begin{equation}
	\sum b_k z_k = z_{00}\sum b_k  + \sum b_k \xi_k \frac{\partial z}{\partial x} + \sum b_k \eta_k \frac{\partial z}{\partial x} \
	+ \frac{1}{2} \sum b_k \xi^2_k \frac{\partial^2 z}{\partial x^2} + \sum b_k \xi_k \eta_k \frac{\partial^2 z}{\partial x \partial y} \
	+ \frac{1}{2} \sum b_k \eta^2_k \frac{\partial^2 z}{\partial y^2}.
\end{equation}
If the $b_k$ can be chosen so that
\begin{equation}
\sum b_k \xi_k = \sum b_k \eta_k = \sum b_k \xi_b \eta_k = 0, \quad \sum b_k \xi^2_k = \sum b_k \eta^2_k = 2,
\label{eq:define}
\end{equation}
then (since all derivatives are evaluated at the central node and thus independent of $k$) we find
\begin{equation}
	\sum b_k z_k = z_{00}\sum b_k  + \frac{\partial^2 z}{\partial x^2} + \frac{\partial^2 z}{\partial y^2}.
\end{equation}
which we use to obtain the curvature at the origin:
\begin{equation}
	\nabla^2 z = \sum b_k z_k - z_{00}\sum b_k.
\end{equation}
A data constraint E in the first quadrant has relative fractional coordinates ($u_E, v_E$).  This means
we obtained these coordinates via
\begin{equation}
	u_E = \frac{x_e - x_0}{\Delta x} = \xi_E, \quad v_E = \frac{y_e - y_0}{\Delta y} = \alpha \eta_E.
\end{equation}
The other four points required are
nodes on the grid and these have relative, fractional $u$- and $v$-coordinates belonging to the set \{$-1, 0, +1$\},
depending on which node we consider. The five equations in (\ref{eq:define}) results in a 5x5 linear system $\mathbf{Mb} = \mathbf{r}$.
The four nodes used in the construction for quadrant 1 are labeled \{NW, W1, S1, SE\}
and reflect their positions relative to the central (0,0) node.
The relative coordinates of these four points populate the first four columns in the matrix $\mathbf{M}$, while point E
contributes to the 5th column. Hence, the matrix equation
for $b_k$ becomes
\begin{equation}
\left[ {\begin{array}{*{20}{r}}
{ - 1}&{ - 1}&0&1&{{u_E}}\\
1&0&{ - 1}&{ - 1}&{{v_E}}\\
1&1&0&1&{u_E^2}\\
{ - 1}&0&0&{ - 1}&{{u_E}{v_E}}\\
1&0&1&1&{v_E^2}
\end{array}} \right] \cdot \left[ {\begin{array}{*{20}{c}}
{{b_1}}\\
{{b_2}}\\
{{b_3}}\\
{{b_4}}\\
{{b_5}}
\end{array}} \right] = \left[ {\begin{array}{*{20}{c}}
0\\
0\\
2\\
0\\
2\alpha^2
\end{array}} \right].
\end{equation}
\PDFfig[h]{off_surface_nodes_rot}{Rotating quadrants 2--4 yields cases that matches the first quadrant,
allowing us to select which nodes play the roles of A-D and what the coordinates of E should be.}{5in}

Given $u_E = \xi$ and $v_E = \alpha \eta$ this matrix equation is identical to SW90. 
Introducing $\Delta = \left (u_E + v_E + 1 \right)\left (u_E + v_E\right)$ we find
the solution using MATLAB's symbolic solver to be
\begin{equation}
\left[ {\begin{array}{*{20}{c}}
{{b_1}}\\
{{b_2}}\\
{{b_3}}\\
{{b_4}}\\
{{b_5}}
\end{array}} \right] = \frac{1}{\Delta }\left[ {\begin{array}{*{20}{c}}
{\alpha^2(u_E^2 + 2{u_E}{v_E} + {u_E}) - (v_E^2 + {v_E})}\\[4pt]
{2\left( {{v_E} - \alpha^2{u_E} + \alpha^2} \right)\left( {{u_E} + {v_E}} \right)}\\[4pt]
{2\left( {\alpha^2{u_E} - {v_E} + 1} \right)\left( {{u_E} + {v_E}} \right)}\\[4pt]
{v_E^2 + {v_E} + 2{u_E}{v_E} - \alpha^2(u_E^2 + {u_E})}\\[4pt]
2(1+\alpha^2)
\end{array}} \right].
\label{eq:new}
\end{equation}
With $\mathbf{M}$ hardwired for quadrant 1 we must rotate other quadrant cases to coincide with the
node geometry of quadrant 1.  Figure~\ref{fig:off_surface_nodes_rot} shows which nodes must serve as
the four surrounding nodes A--D.  Also, it is evident that we must also take care to
\begin{enumerate}
	\item Use the absolute values of $u_E$ and $v_E$.
	\item For quadrants 2 and 4 we must swap $u_E$ and $v_E$.
\end{enumerate}
With these considerations the $\mathbf{Mb = r}$ solution for $\mathbf{b}$ can be used for all quadrants.

In SW90 we derived the solution for $\mathbf{b}$ as
\begin{equation}
\left[ {\begin{array}{*{20}{c}}
{{b_5}}\\
{{b_4}}\\
{{b_3}}\\
{{b_2}}\\
{{b_1}}
\end{array}} \right] = \left[ {\begin{array}{*{20}{c}}
{\frac{2(1+\alpha^2)}{\left (u_E + v_E + 1 \right)\left (u_E + v_E\right)}}\\[4pt]
{1 - \frac{1}{2}b_5u_E\left(1+u_E\right)}\\[4pt]
{v_E \left(1+u_E\right)b_5 - 2b_4}\\[4pt]
{\left(u_E+v_E\right)b_5 - b_3}\\[4pt]
{u_Eb_5 + b_4 - b_2}
\end{array}} \right]
\label{eq:paper}
\end{equation}
which includes a correction to $b_2$ due to a typesetting error.  When we insert an arbitrary first-quadrant
$(u_E, v_E)$ position for any value of $\alpha$ into (\ref{eq:new}) and (\ref{eq:paper}) they give identical results.
We will therefore continue to use the original solution as it is numerically faster.
However, the original surface implementation does \emph{not} honor the swapping of $u_E$ and $v_E$ for quadrants
2 and 4.  This means the current surface implementation introduces subtle distortions in the grid at about half
the nodes with data constraints.

\section{Determine $C_{00}$ using interpolating biquadratic}

SW90 used a 2-D Taylor expansion from the center node to the off-node data constraint E
and 4 other nodes in order to include data constraints into the finite difference solution.
An alternative approach to incorporating off-node data constraints is fitting a biquadratic model
\begin{equation}
	z(u,v) = c_1 + c_2 u + c_3 v + c_4 u v + c_5 u^2 + c_6 v^2 + c_7 u^2v + c_8 v^2 u + c_9 u^2v^2
	\label{eq:biquadratic}
\end{equation}
to the nearest 8 nodes plus the single data constraint in the vicinity of ($0,0$).  Using the nodes 
\{NW, N1, NE, W1, E1, SW, S1, SE \} relative to the center node we obtain the following matrix equation
for the 9 unknown coefficients $c_k$ defining the exact interpolation for quadrant 1:
\begin{equation}
\left[ {\begin{array}{*{20}{c}}
1&{ - 1}&1&{ - 1}&1&1&1&{ - 1}&1\\
1&0&1&0&0&1&0&0&0\\
1&1&1&1&1&1&1&1&1\\
1&{ - 1}&0&0&1&0&0&0&0\\
1&1&0&0&1&0&0&0&0\\
1&{ - 1}&{ - 1}&1&1&1&{ - 1}&{ - 1}&1\\
1&0&{ - 1}&0&0&1&0&0&0\\
1&1&{ - 1}&{ - 1}&1&1&{ - 1}&1&1\\
1&u&v&{uv}&{{u^2}}&{{v^2}}&{{u^2}v}&{{v^2}u}&{{u^2}{v^2}}
\end{array}} \right] \cdot \left[ {\begin{array}{*{20}{c}}
{{c_1}}\\
{{c_2}}\\
{{c_3}}\\
{{c_4}}\\
{{c_5}}\\
{{c_6}}\\
{{c_7}}\\
{{c_8}}\\
{{c_9}}
\end{array}} \right] = \left[ {\begin{array}{*{20}{c}}
{{z_1}}\\
{{z_2}}\\
{{z_3}}\\
{{z_4}}\\
{{z_5}}\\
{{z_6}}\\
{{z_7}}\\
{{z_8}}\\
w
\end{array}} \right].
\end{equation}
Here, $z_1$ -- $z_8$ are the node values for the selected 8 grid nodes above;
these all have relative $u$ and $v$ coordinates in the \{$-1, 0, +1$\} set.
The single off-node data constraint is given by ($u,v,w$).  We only need
(\ref{eq:biquadratic}) to evaluate $C_{00} = \nabla^2 z \rvert _{0, 0} = 2 \left(c_5 + c_6 \right)$
as all other terms are zero at the origin.  This expression relates the curvature of the
surface at the center node to our off-node data point.  Letting MATLAB do the algebra,
we obtain
\begin{equation}
C_{00} = \frac{D}{(u^2-1)(v^2-1)}
\label{eq:biquad}
\end{equation}
with
\begin{equation}
\begin{split}
D = z_2 + z_4 + z_5 + z_7 - 4w - 2u z_4 + 2 u z_5 + 2 v z_2 - 2 v z_7 - u^2 z_2   \\
    + u^2 z_4 + u^2 z_5 - u^2 z_7 + v^2 z_2 - v^2 z_4 - v^2 z_5 + v^2 z_7 - u v^2 z_1    \\
    + u^2 v z_1 - 2 u^2 v z_2 + u v^2 z_3 + u^2 v z_3 + 2 u v^2 z_4 - 2 u v^2 z_5 - u v^2 z_6  \\
    - u^2 v z_6 + 2 u^2 v z_7 + u v^2 z_8 - u^2 v z_8 + u^2 v^2 z_1 - u^2 v^2 z_2 + u^2 v^2 z_3   \\
    - u^2 v^2 z_4 - u^2 v^2 z_5+ u^2 v^2 z_6 - u^2 v^2 z_7 + u^2 v^2 z_8 - u v z_1 + u v z_3 \\
    + u v z_6 - u v z_8.
\end{split}
\end{equation}
which we can reorganize to
\begin{equation}
\begin{split}
D = (z_2 + z_4 + z_5 + z_7 - 4w)   \\
	+ 2u (z_5 - z_4) + 2v (z_2 - z_7) + u v (z_3 + z_6 - z_1 - z_8) \\
	- u^2 (z_4 + z_5 - z_2 - z_7) + v^2 (z_2 - z_4 - z_5 + z_7)    \\
	+ u^2 v (z_1 - 2 z_2 + z_3 - z_6 + 2 z_7 - z_8) \\
	- u v^2 (z_1 + z_3 + 2 z_4 - 2 z_5 - z_6 + z_8)  \\
	+ u^2 v^2 (z_1 - z_2 + z_3 - z_4 - z_5 + z_6 - z_7 + z_8).
\end{split}
\end{equation}
From SW90 we know the desired surface must satisfy
\begin{equation}
\left( 1 - T_I\right)\nabla^4 z - T_I \nabla^2 z = \left( 1 - T_I\right)\nabla^2 C - T_I \nabla^2 z = 0.
\label{eq:biharm}
\end{equation}
We also derived
\begin{equation}
\nabla^2 C \rvert _{0, 0} = C_{10} + C_{-10} + \alpha^2(C_{01} + C_{0-1}) - 2 (1+\alpha^2)C_{00}.
\label{eq:curv}
\end{equation}
and
\begin{equation}
C_{00} = \nabla^2 z \rvert _{0, 0} = z_{10} + z_{-10} + \alpha^2(z_{01} + z_{0-1}) - 2 (1+\alpha^2)z_{00}.
\end{equation}
We will insert (\ref{eq:curv}) into (\ref{eq:biharm}) and replace $C_{00}$ with the value obtained from (\ref{eq:biquad}), then algebraically solve for $z_{00}$.
This gives
\begin{equation}
\begin{split}
z_{00} = \frac{1}{2(1+\alpha^4)} \{ 2(1+\alpha^2) \left [ z_{10}+z_{-10}+\alpha^2(z_{01}+z_{0-1})\right ] \\
	-2\alpha^2\left [z_{11} + z_{1-1} + z_{-11} + z_{-1-1}\right ] - \left [z_{20} + z_{-20} + \alpha^4 (z_{02} + z_{0-2})\right ] \\
	+ \left (\frac{T_I}{1-T_I} + 2(1+\alpha^2)\right )C_{00} \}.
\label{eq:offnode}
\end{split}
\end{equation}
Thus, we will use (\ref{eq:offnode}) instead of (A-7) when updating nodes with nearby data constraints.

\end{document}
