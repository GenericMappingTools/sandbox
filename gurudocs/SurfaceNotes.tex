%	$Id$
% Internal documentation for SOliving Briggs coefficients
% Paul Wessel, April 21, 2015.
\documentclass[12pt,letterpaper,margin=0.5in]{article}
\usepackage{times}
\usepackage{epsfig}
\usepackage{graphicx}
\usepackage{breqn}
\usepackage[margin=1in]{geometry}
\newcommand{\PDFfig}[4][tbp]{\begin{figure}[#1] \centering \epsfig{figure=#2,width=#4} \caption{{\small #3}} \label{fig:#2} \end{figure}}

\begin{document}

\section{THE SURFACE ALGORITHM}

The 1990 Smith and Wessel Geophysics article (here SW90) discusses the
algorithm we use to perform gridding with continuous curvature splines in tension.
This document discusses some of the nitty-gritty details that are used in the
implementation of surface.c. This implementation was written during GMT 1 days
and used a Fortran west-to-east column arrangement with rows increasing upwards.
Since GMT 2 we have been using a scanline arrangements with rows arranged from north
to south and each column marches from west to east.  Thus, we would like to recast
the surface algorithm to use this node arrangement to avoid having to transpose
the matrix at the end, and to take advantage of the GMT library functions that
all expect scanline orientation.  Sections below discusses various aspects of the
algorithm and at times proposed improvements.

\section{BRIGGS COEFFICIENT SOLUTIONS}

\PDFfig[h]{off_surface_nodes}{The four quadrants (1-4) where a data constraint (E) may be located.  The Briggs
solution published in SW90 is defined for the first quadrant (pink) only.  Other quadrants must be rotated
so that the point E falls into the first quadrant and A--D can be identified.  Black square is the central node (0,0).}{3in}

Most data constraints are likely not to fall exactly (or even very close) to the
desired output nodes and we must find a way to include them in the finite difference
solution.  Figure~\ref{fig:off_surface_nodes} shows the four cases we must consider since
a point may fall into any of the four quadrants surrounding our central node.
We have followed Briggs [1974] in connecting the off-node point via a
second-order Taylor expansion from the central node ($z_{00}$) to a nearby point $z_k$:
\begin{equation}
	z_k = z_{00} + \xi_k \frac{\partial z}{\partial x} + \eta_k \frac{\partial z}{\partial x} \
	+ \frac{1}{2} \xi^2_k \frac{\partial^2 z}{\partial x^2} + \xi_k \eta_k \frac{\partial^2 z}{\partial x \partial y} \
	+ \frac{1}{2}  \eta^2_k \frac{\partial^2 z}{\partial y^2}.
\end{equation}
Here, the coordinates of the off-node points ($\xi_k, \eta_k$) are defined on a normalized
grid where the grid cell of dimension $\Delta x$ by $\Delta y$ equals 1x1, i.e.,
\begin{equation}
\xi_k = \frac{x_k-x_{00}}{\Delta x} \quad \alpha \eta_k = \frac{y_k-y_{00}}{\Delta y} = \frac{y_k-y_{00}}{\Delta x},
\end{equation}
where $\alpha = \Delta y/ \Delta x$ is the grid spacing anisotropy.
The Briggs scheme is to evaluate this expansion to the five distinct points A--E, here identified as $(\xi_k, \eta_k), k = 1,5$.
We then multiply each expansion by a real number $b_k$ and sum the five expressions:
\begin{equation}
	\sum b_k z_k = z_{00}\sum b_k  + \sum b_k \xi_k \frac{\partial z}{\partial x} + \sum b_k \eta_k \frac{\partial z}{\partial x} \
	+ \frac{1}{2} \sum b_k \xi^2_k \frac{\partial^2 z}{\partial x^2} + \sum b_k \xi_k \eta_k \frac{\partial^2 z}{\partial x \partial y} \
	+ \frac{1}{2} \sum b_k \eta^2_k \frac{\partial^2 z}{\partial y^2}.
\end{equation}
If the $b_k$ can be chosen so that
\begin{equation}
\sum b_k \xi_k = \sum b_k \eta_k = \sum b_k \xi_b \eta_k = 0, \quad \sum b_k \xi^2_k = \sum b_k \eta^2_k = 2,
\label{eq:define}
\end{equation}
then (since all derivatives are evaluated at the central node and thus independent of $k$) we find
\begin{equation}
	\sum b_k z_k = z_{00}\sum b_k  + \frac{\partial^2 z}{\partial x^2} + \frac{\partial^2 z}{\partial y^2}.
\end{equation}
which we use to obtain the curvature at the origin:
\begin{equation}
	\nabla^2 z = \sum b_k z_k - z_{00}\sum b_k.
\end{equation}
A data constraint E in the first quadrant has relative fractional coordinates ($u_E, v_E$).  This means
we obtained these coordinates via
\begin{equation}
	u_E = \frac{x_e - x_0}{\Delta x} = \xi_E, \quad v_E = \frac{y_e - y_0}{\Delta y} = \alpha \eta_E.
\end{equation}
The other four points required are
nodes on the grid and these have relative, fractional $x$- and $y$-coordinates belonging to the set \{$-1, 0, +1$\},
depending on which node we consider. The five equations in (\ref{eq:define}) results in a 5x5 linear system $\mathbf{Mb} = \mathbf{r}$.
The four nodes used in the construction for quadrant 1 are labeled \{NW, W1, S1, SE\}
and reflect their positions relative to the central (0,0) node.
The coordinates of these four points populate the first four columns in the matrix $\mathbf{M}$, while point E
contributes to the 5th column. Hence, the matrix equation
for $b_k$ becomes
\begin{equation}
\left[ {\begin{array}{*{20}{r}}
{ - 1}&{ - 1}&0&1&{{u_E}}\\
1&0&{ - 1}&{ - 1}&{{v_E}}\\
1&1&0&1&{u_E^2}\\
{ - 1}&0&0&{ - 1}&{{u_E}{v_E}}\\
1&0&1&1&{v_E^2}
\end{array}} \right] \cdot \left[ {\begin{array}{*{20}{c}}
{{b_1}}\\
{{b_2}}\\
{{b_3}}\\
{{b_4}}\\
{{b_5}}
\end{array}} \right] = \left[ {\begin{array}{*{20}{c}}
0\\
0\\
2\\
0\\
2\alpha^2
\end{array}} \right].
\end{equation}
Given $u_E = \xi$ and $v_E = \alpha \eta$ this matrix equation is identical to what we published in SW90. 
Introducing $\Delta = \left (u_E + v_E + 1 \right)\left (u_E + v_E\right)$ we find
the solution
\begin{equation}
\left[ {\begin{array}{*{20}{c}}
{{b_1}}\\
{{b_2}}\\
{{b_3}}\\
{{b_4}}\\
{{b_5}}
\end{array}} \right] = \frac{1}{\Delta }\left[ {\begin{array}{*{20}{c}}
{\alpha^2(u_E^2 + 2{u_E}{v_E} + {u_E}) - (v_E^2 + {v_E})}\\[4pt]
{2\left( {{v_E} - \alpha^2{u_E} + \alpha^2} \right)\left( {{u_E} + {v_E}} \right)}\\[4pt]
{2\left( {\alpha^2{u_E} - {v_E} + 1} \right)\left( {{u_E} + {v_E}} \right)}\\[4pt]
{v_E^2 + {v_E} + 2{u_E}{v_E} - \alpha^2(u_E^2 + {u_E})}\\[4pt]
2(1+\alpha^2)
\end{array}} \right]
\label{eq:upper}
\end{equation}
With $\mathbf{M}$ hardwired for quadrant one we must rotate any other quadrant case to coincide with the
node geometry of quadrant 1.  Figure~\ref{fig:off_surface_nodes_rot} shows which nodes must serve as
the four surrounding nodes A--D.  Also, note that we must also take care to
\PDFfig[h]{off_surface_nodes_rot}{Rotating quadrants 2--4 yields cases that matches the first quadrant,
allowing us to select which nodes play the roles of A-D and what the coordinates of E should be.}{3in}
\begin{enumerate}
	\item Use the absolute values of $u_E$ and $v_E$.
	\item For quadrants 2 and 4 we must swap $u_E$ and $v_E$.
\end{enumerate}
With these considerations the $\mathbf{Mb = r}$ can be used for all quadrants.
\end{document}
