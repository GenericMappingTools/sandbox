%	$Id$
% Internal documentation for filling wrapped polygons
% Paul Wessel, May 30, 2018.
\documentclass[12pt,letterpaper,margin=0.5in]{article}
\usepackage{times}
\usepackage{epsfig}
\usepackage{amsmath}
\usepackage{graphicx}
\usepackage{breqn}
\usepackage[margin=1in]{geometry}
\newcommand{\PDFfig}[4][tbp]{\begin{figure}[#1] \centering \epsfig{figure=#2,width=#4} \caption{{\small #3}} \label{fig:#2} \end{figure}}
\pagenumbering{arabic}

\begin{document}
\section{WRAPPED POLYGON FILL}

In GMT, user-provided polygons are plotted using modules \texttt{psxy} and \texttt{psxyz}, while landmass polygons are indirectly plotted by \texttt{pscoast}.
The latter is different and does not plot large geographic polygons directly.  Instead, polygons in a global database (GSHHG) have been drawn and
quartered earlier and the final polygon is reassembled as a series of tiles that combine to yield a filled large
polygon.  Any outline can then be drawn separately since we do not wish to draw outlines for individual tiles as
this would include interior tile boundaries.  However, \texttt{psxy} and \texttt{psxyz} may be given large geographical
polygons and there is no tiling-machinery there to do the job as done in \texttt{pscoast}.  Hence, bad things
can happen if we are not careful.  This document discusses our approach to this problem.  It is only a problem
for geographic data and when a map projection is chosen that results in a particular meridian (typically at 180 degrees
away from the central meridian) actually appears twice in the map.  We call this meridian a \emph{periodic map boundary}.
Unlike a spherical globe, this means a polygon that straddles this boundary will exit at one side of the map and reappear
at the other side, typically resulting in two parts of a polygon where only one was provided.
GMT currently handles large geographic polygons that may exit and reenter at periodic boundaries by a series of steps:
\begin{enumerate}
	\item Project the $(\lambda, \phi)$ coordinates to plot $(x,y)$ inches on a map.
	\item Determine if the $(x,y)$ line crosses the periodic map boundary and insert a new point at the crossing at both the exit
		and reentry locations. These points have the same $y$ coordinate but differ in their $x$ values by the
		\emph{width} of the map at that latitude.
	\item Detect these horizontal jumps across the map and flag them with ``pen up'' instructions (we use 1 for move, 0 for draw)
		after reaching the boundary.  Hence, most such $(x,y)$ arrays starts with a pen of 1 (gotta start drawing somewhere)
		and then are followed by a series of draws (pen = 0) until we reach the boundary.  The duplicated point on the
		opposite boundary therefore will have pen up (pen = 1) so that we do not draw a horizontal line across the map (this
		is how we can draw lines. However, to fill the polygon we cannot use this path).
	\item Undo the implicit wrapping and truncate the polygon against both the left and the right periodic boundaries.
		This step involves adding portions of the map border to build a temporary polygon.  We then plot fill this polygon.
		In fact, the GMT algorithm does this three times to cover some complicated cases no longer documented.
	\item If an outline is desired then this is drawn separately at the end using the pen up/down information discussed earlier.
\end{enumerate}
This algorithm is complicated, not documented to be understandable, and remains error-prone, as numerous bug reports from EarthByte
can testify to (they end up with landmasses that have been reconstructed back in time and these reconstructed polygons provide many test
cases that at times are challenging to plot correctly when filled).  We especially find polygons that are rotated to become \emph{polar caps},
i.e., they end up enclosing one of the geographic poles, leading to additional challenges.

\section{PROPOSED SIMPLER ALGORITHM}
We are considering simplifications that will make the algorithm simpler and easier to understand (and hopefully foolproof).
\subsection{Global Map: Vertical periodic boundaries}

\PDFfig[h]{pwrap_JM_1}{A large polygon that straddles Greenwich.  Mercator projection.}{5in}
A test polygon that crosses back and forth over Greenwich is shown in Figure~\ref{fig:pwrap_JM_1}.  With Greenwich as the central meridian we have a simple
polygon to fill.  However, if we shift the central meridian to 180 then the same polygon still crosses Greenwich which
now appears as our periodic east and west map boundaries. Simply projecting the coordinates and connecting the dots yields a wrapped polygon
with nasty horizontal jumps back and forth across the map
(Figure~\ref{fig:pwrap_JM_2}).  The proposed new algorithm would simply use the information of jumps and add or
subtract the map width to the $x$ coordinates until we again reenter the border.  If we do this for both periodic
boundaries we obtain two polygons that both are partly inside the (projected) map area (Figure~\ref{fig:pwrap_JM_3}).
However, since we always turn on \emph{map clipping} using the boundaries of the map then we can fill these polygons and
only the parts that are inside the map region will be visible (Figure~\ref{fig:pwrap_JM_4}). We note that in the case of straight map
boundaries the map width $W_{360}$ is constant across all latitudes and the two polygons in Figure~\ref{fig:pwrap_JM_3} are identical
except for a translation of $W_{360}$.  Thus, in this case we could simply write the $(x,y)$ coordinates once as a named \emph{PostScript} array,
plot it once, translate the origin by $W_{360}$, and plot the same array again.
\PDFfig[h]{pwrap_JM_2}{Same polygon naively drawn for a central meridian 180 degrees away.  Individual points are shown
with a red dots while a thin line connects the dots as given.  Mercator projection.}{5in}
\PDFfig[h]{pwrap_JM_3}{The projected polygon wrapped to left and right to avoid jumps, thus exceeding the map domain.  Mercator projection.}{5in}
\PDFfig[h]{pwrap_JM_4}{The same polygons being filled, subject to map region clipping.  Mercator projection.}{5in}

\subsection{Global Map: Curved periodic boundaries}

\PDFfig[h]{pwrap_JH_1}{A large polygon that straddles Greenwich.  Hammer projection.}{5in}
For many other map projections the periodic boundaries will not be straight (they are ellipses or more complicated curves).  In this case the
map width is a \emph{function} of latitude or, equivalently, the projected $y$ coordinate, i.e., $W_{360}(y)$.  The same polygon we used before is now plotted
using the Hammer projection in Figure~\ref{fig:pwrap_JH_1}.  We follow exactly the same steps as used for the Mercator projection,
with the exception that the map width we add or subtract is dependent on the $y$ value at the jump. Figures~\ref{fig:pwrap_JM_2},
\ref{fig:pwrap_JM_3}, and \ref{fig:pwrap_JM_4} show the comparable results.  As can be seen, this algorithm is general and works
equally well for straight or curved boundaries.  However, there is one important difference: The left and right polygons are no longer of identical shape,
hence we cannot use a saved \emph{PostScript} array twice but must instead write the two polygons separately.
\PDFfig[h]{pwrap_JH_2}{Same polygon naively drawn for a central meridian 180 degrees away.  Individual points are shown
with a red dots while a thin line connects the dots as given.  Hammer projection.}{5in}
\PDFfig[h]{pwrap_JH_3}{The projected polygon wrapped to left and right to avoid jumps, thus exceeding the map domain
Note the left is no longer an identical copy of the right polygon. Hammer projection.}{5in}
\PDFfig[h]{pwrap_JH_4}{The same polygons being filled, subject to map region clipping.  Hammer projection.}{5in}

\subsection{Polygons with holes}
For donut polygons we must treat all holes to the same procedure we use on the perimeter, regardless of whether or not they cross the periodic boundary.
The holes must always be present so that the``even/odd-clip'' algorithm can determine the correct polygon inside to be filled.

\subsection{Polar caps}
Polar caps are by definition crossing a periodic boundary since they have a 360-degree range in longitudes.  Such polygons are the
bane of most GIS software (such as arcGIS or Google Earth) since they fail to give a filled polygon.  Instead, the polygon must be split into
``east'' and ``west'' polygons so that their fill algorithm works.  GMT hides this complexity under the hood and does not require the users to
split their polygons manually.  When a polar cap is detected for a projection with periodic boundaries then another algorithm is applied.  Here,
we must stitch in part of the map boundary, which may be curved or straight, in order to create a fillable polygon.  A simpler mechanism is proposed here:
Given the $(x,y)$ and pen array, create the fillable polygon by following the line, and when exiting the map border we append another point with
the same $x$ value and a $y$-value that is either 0 (for south polar caps) or $H$ (the map height, for north polar caps). When we reenter on the
other side we \emph{prepend} a similar point, and continue appending the actual points of the line.  Given map clipping is in effect the extrusions
toward the top of bottom of the map will be clipped at the boundary.

\subsection{Oblique domains or less than 360-degree longitudinal extent}
GMT can specify regions using the lower left and upper right corners of a rectangle, thus yielding a rectangular map region despite these
borders no longer represent meridians and parallels.  Likewise, users may specify a subset of longitudes (and latitudes) so that there are
no periodic borders visible on the map.  In this case the map may be rectangular or have curved borders.  Nevertheless, any polygon that
may cross such invisible borders (i.e., they would often be outside the frame)
will still lead to jumps and should receive exactly the same treatment as the global maps, with one revision:  As the jumps will occur when we
pass 180 degrees from the central meridian and jump to the other side of the map, that jump is no longer equal to the actual map width $W(y)$ but is given by the
\emph{equivalent map width} we would have if the full global region had been specified.  Hence, the algorithm will still check for jumps and if detected we follow
the above algorithm but using $W_{360}(y)$ instead of $W(y)$.  The limited map region clipping path will restrict plotting to its domain anyway.
If is likely that for many such domains one of the translated polygons may be entirely outside the map domain, in which case we do not even need to
plot it.  I note that we currently have a hard time with plotting polygons with large longitudinal extent on a relative small map region because the
polygon part far outside the map region, when projected, will wrap to the other side and result in horizontal lines on the map.

\section{SUMMARY}

I propose the following outline for a new \texttt{gmt\_plot_polygons} function:
\begin{itemize}
	\item If Cartesian data then simply convert to map $(x,y)$ and pass to \texttt{PSL\_plotpolygon} directly.
	\item No longer clip the polygon coordinates to the (complex) map border.  Instead,
	 just call \texttt{gmt\_geo\_to\_xy\_line} which returns the $(x,y,p)$ arrays, with
	$p$ the pen up/down codes.  The length of these arrays may be longer than the input
	$(\lambda, \phi)$ array by virtue of added wrap points. It is this array (and pen code)
	that we (successfully) use to draw the outline of any polygon (clipped or not).
	\item Determine if polygon is wrapping or not.  If not wrapping the fill and outline as requested.
	\item We have a wrapping polygon.
\end{itemize}
\end{document}
