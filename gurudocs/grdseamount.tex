%------------------------------------------
% Notes regarding grdflexure firmovisous response function
% Paul Wessel, May 2020

\documentclass[12pt]{article}
\usepackage{amsmath}
\usepackage{epsfig}
\usepackage{makeidx}
\usepackage{float}
\usepackage{times}
\usepackage{mathptm}

\begin{document}
%------------------------------------------

%------------------------------------------
{\center \bf Amplitude calculations in grdseamount (8/1/2020)}\\

The {\bf grdseamount} module can compute the amplitude $h(\mathbf{x,t})$ of a growing seamount
for four different types of shape
\begin{enumerate}
	\item A cone
	\item A paraboloid
	\item A Gaussian
	\item A disc
\end{enumerate}
Except for the disc, all shapes may have a truncated top determined by a flattening parameter $f$
that goes from 0 (no flattening) to 1 (no seamount at all).
All shapes may have an elliptical or circular basal area.  Given a normalized volume fraction function
$v(t)$ that goes from 0 to 1 over the lifespan of the seamount, we wish to determine the
parameters of the shape that yields the actual volume at time $t$, which is simply
\begin{equation}
V(t) = V_0 v(t),
\label{eq:vt}
\end{equation}
where $V_0$ is the final volume of the fully grown seamount of the given shape. To yield a unique result, we
relate the radius $r(t)$ and height $h(t)$ at a given time to a constant slope, so height is a function
of the radius.  With this conformality we can solve for the radius (and major, minor axes in
the elliptical case).  Below, we define $r_0$ as either the final circular radius or the final
major axis ($a$), with $e = b/a$ as the eccentricity (which is 1 for circular). In all cases,
the minor axis is simply $b(t) = e a(t)$, and $h^*$ is the projected height in the absence of truncation.

{\center \bf 1. Growing conical seamount}\\

We wish to determine $r(t)$ from $V(t)$.  First, the final volume is given by
\begin{equation}
V_0 = \frac{e \pi \alpha (1 - f^3)}{3}r_0^3,
\end{equation}
where the constant slope of the cone is given by
\begin{equation}
\alpha = \frac{h^*}{r_0} = \frac{h_0}{(1-f)r_0} = \frac{h(t)}{(1-f)r(t)}.
\end{equation}
We then wish to solve
\begin{equation}
V(t) = \frac{e \pi \alpha (1 - f^3)}{3}r(t)^3,
\end{equation}
yielding
\begin{equation}
r(t) = \left ( \frac{3 V(t)}{e \pi \alpha (1 - f^3)} \right )^{1/3}.
\end{equation}
This then yields
\begin{equation}
h(t) = \alpha (1 - f) r(t).
\end{equation}

{\center \bf 1. Growing paraboloid}\\

We wish to determine $r(t)$ from $V(t)$.  First, the final volume is given by
\begin{equation}
V_0 = \frac{e \pi \alpha (1 - f^4)}{2}r_0^3,
\end{equation}
where the constant slope of the paraboloid is given by
\begin{equation}
\alpha = \frac{h^*}{r_0} = \frac{h_0}{(1-f^2)r_0} = \frac{h(t)}{(1-f^2)r(t)}.
\end{equation}
We then wish to solve
\begin{equation}
V(t) = \frac{e \pi \alpha (1 - f^4)}{2}r(t)^3,
\end{equation}
yielding
\begin{equation}
r(t) = \left ( \frac{2 V(t)}{e \pi \alpha (1 - f^4)} \right)^{1/3}.
\end{equation}
This then yields
\begin{equation}
h(t) = \alpha (1 - f^2) r(t).
\end{equation}


{\center \bf 1. Growing Gaussian}\\

We wish to determine $r(t)$ from $V(t)$.  Because of truncation, we need to compute the height scale
\begin{equation}
\gamma = e^\frac{9}{2}f^2.
\end{equation}
which of course is 1 if no truncation, and
where the constant slope of the Gaussian is given by
\begin{equation}
\alpha = \frac{h^*}{r_0} = \frac{h_0}{\gamma r_0} = \frac{h(t) \gamma}{r(t)}.
\end{equation}
First, the final volume is given by
\begin{equation}
V_0 = \frac{2 e \pi \alpha (1 + \frac{9}{2}f^2)}{9\gamma}r_0^3.
\end{equation}
We then wish to solve
\begin{equation}
V(t) = \frac{e \pi \alpha (1 + \frac{9}{2}f^2)}{9\gamma}r(t)^3,
\end{equation}
yielding
\begin{equation}
r(t) = \left ( \frac{9 \gamma V(t)}{2 e \pi \alpha (1 + \frac{9}{2}f^2))} \right)^{1/3}.
\end{equation}
This then yields
\begin{equation}
h(t) = \frac{\alpha}{\gamma} r(t).
\end{equation}


In all cases, inserting (\ref{eq:vt}) into the expressions for $r(t)$ gives
\begin{equation}
r(t) = v(t)^{1/3} r_0.
\end{equation}

\end{document}
