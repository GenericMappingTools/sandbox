%	$Id$
% Internal documentation for slide modeling in grdseamount
% Paul Wessel, November 16, 2016.
\documentclass[12pt,letterpaper,margin=0.5in]{report}
\usepackage{times}
\usepackage{graphicx}
\usepackage{breqn}
\usepackage[margin=0.5in]{geometry}
\usepackage{lscape}
\textheight = 9 in
\topmargin = -1 in
\begin{document}

\section{Modeling landslides in grdseamount}

\begin{figure}[h!]
  \centering
  \includegraphics[width=5in]{slides_model.png}
  \caption{Geometry for an \emph{ad hoc} landslide approximation in \texttt{grdseamount}.  The slide material (pink)
  will be deposited at the toe of the seamount (light blue) starting at a height of $h_c$ and linearly
  tapering to zero at a distal point $r_d$. Note that $h_2 > h_1$ while $r_1 > r_2$.}
  \label{slides_model}
\end{figure}

We expand the work of {\it J. Smith and Wessel} (2000) on approximating large landslides off seamounts
for the purpose of studying the isostatic consequences of the mass redistribution (Figure~\ref{slides_model}).  We model the
cross-sectional shape of the slide over the radial range $r_2$ to $r_1$ by
\begin{equation}
h_s(r) = h_1 + \Delta h \cdot q(r),
\end{equation}
where $\Delta h = h_2 - h_1$ is the \emph{escarpment} height of the rupture and $q(u)$ is the normalized shape curve of the slide,
\begin{equation}
q(u) = u_0 \left (\frac{1 + u_0}{u + u_0} - 1 \right ),
\end{equation}
i.e., a hyperbolic curve.  Here, $u$ is the normalized distance from $r_2$ to $r_1$, given by
\begin{equation}
u = \frac{r-r_2}{r_1 - r_2} = \frac{r-r_2}{\Delta r}, \quad \mbox{hence } r = r_2 + u \Delta r,
\end{equation}
where $\Delta r$ is the \emph{radial} range of the slide. We modify the parameter $u_0$ to carve more or less deeply into the seamount core.
Figure~\ref{slides_shape} shows typical shapes of $q(u)$ for some values of $u_0$.
\begin{figure}[h!]
  \centering
  \includegraphics[width=5in]{slides_shape.png}
  \caption{A variety of slide shapes are possible by varying $u_0$.  The slide area for a conical seamount would be the area
  between the flank (dashed line) and the selected curve. A smaller $u_0$ will cut more deeply into the seamount.}
  \label{slides_shape}
\end{figure}

Depending on the shape of the seamount prior to the slide event (called $h(r)$, with $h_s(r) \le h(r)$), we will need to compute the various radii referred
to in Figure \ref{slides_model}, such as $r_1$, $r_2$, $r_c$, and $r_d$ from their corresponding heights $h_1$, $h_2$, $h_c$ and volume of slide.
The material removed by the land slide (pink)
will be deposited at the base of the seamount (light blue) starting from the proximal height $h_c$.  We assume this deposit will have a linearly
decaying height, enabling us to compute the distal radius $r_d$ from the volume of the slide, $V_s$.  Clearly, $V_s$ depends
on both $h(r)$ and $h_s(r)$.  We will compute this volume as $V_s = V_f - V_q$, where $V_f$ is the fixed flank volume whose
trapezoidal (for a cone) crossection is delimited by the lines $r = r_2$, $h = 0$ and $h(r)$.  Its area $A_f$ and centroid $\bar{r}_f$ are computed and
we then use Pappas' theorem to get the corresponding volume $V_f = 2 \pi \bar{r}_f A_f$.  For $V_q$, the upper limit is $h_s(r)$ instead. We first compute the upper part of the area, $A^u_q$ (light gray area)
and later we will add the pedestal area $A^l_q$ below $h_1$ (darker gray area):
\begin{equation}
A^u_q = \int_{r_2}^{r_1} \Delta h q(r) dr = \Delta h \Delta r \int_0^1 q(u) du = \Delta h \Delta r u_0 \left [ (1 + u_0) \log \left (\frac{1 + u_0}{u_0} \right ) - 1 \right ].
\end{equation}
To use Pappas' theorem we need the radius to the centroid, obtained via
\begin{equation}
\bar{r}^u_q = \frac{\int_{r_2}^{r_1}q(r)rdr}{\int_{r_2}^{r_1}q(r)dr} = r_2 + \Delta r \bar{u}_q^u = r_2 + \Delta r \frac{\int_0^1q(u)udu}{\int_0^1 q(u)du}.
\end{equation}
This ratio of integrals yields
\begin{equation}
\bar{u}^u_q = \frac{(1 + u_0)\left [1 - u_0 \log \left ( \frac{1+u_0}{u_0} \right ) \right ] - \frac{1}{2}}{(1 + u_0) \log \left (\frac{1 + u_0}{u_0} \right ) - 1}.
\end{equation}
To compute the volume of the pedestal we need its area $A^l_q = \Delta r h_1$ and centroid radius
\begin{equation}
\bar{r}^l_q = \frac{ h_1\int_{r_2}^{r_1} rdr}{A^l_q} = \frac{1}{2} (r_1 + r_2).
\end{equation}
Our final slide volume is therefore
\begin{equation}
V_s = V_f - 2 \pi \left (A^u_q \bar{r}^u_q + A^l_q \bar{r}^l_q \right ),
\end{equation}
yielding
\begin{equation}
V_s = V_f - 2 \pi \Delta r \left \{ \Delta h \left ( r_2 + \Delta r\bar{u}_q^u \right ) u_0 \left [ (1 + u_0) \log \left (\frac{1 + u_0}{u_0} \right ) - 1 \right ] + \frac{h_1}{2} (r_1 + r_2) \right \}.
\label{eq:Vs}
\end{equation}
To solve for the distal radius $r_d$ we need to equate $V_s$ with the equivalent distal volume.
Given that the distal triangle thickness and area\footnote{We assume that $h(r)$ from $h_c$ to zero may be approximated as a linear ramp.} are
\begin{equation}
h_d(r) = h_c \left (1 - \frac{r - r_c}{r_d - r_c}\right ), \quad A_d = \frac{h_c}{2} (r_d - r_0)
\end{equation}
we find the volume to be
\begin{equation}
V_d = 2 \pi \bar{r}_d A_d = 2 \pi \bar{r}_d \left [ \frac{h_c}{2} (r_d - r_0) \right ],
\label{eq:Vd}
\end{equation}
where $\bar{r}_d$ is the centroid radial distance of the distal triangle and is obtained in a similar fashion to $\bar{r}_q^u$ (except
we must handle the awkward initial section separately):
\begin{equation}
\bar{r}_d = \frac{\int_{r_c}^{r_d}h_c \left (1 - \frac{r - r_c}{r_d - r_c} \right )rdr - \int_{r_c}^{r_o}h_c \left (1 - \frac{r - r_c}{r_o- r_c} \right )rdr}{A_d} = \frac{r_c + r_0 + r_d}{3}.
\label{eq:rd}
\end{equation}
Inserting (\ref{eq:rd}) into (\ref{eq:Vd}) and equating it to $V_s$ lead to the quadratic equation
\begin{equation}
r_d^2 + r_c r_d - \left (r_0^2 + r_0 r_c + \frac{3 V_s}{\pi h_c}\right ) = r_d^2 + r_c r_d - c = 0,
\end{equation}
with unique solution
\begin{equation}
r_d = \frac{-r_c + \sqrt{r_c^2 + 4c}}{2}.
\end{equation}

\subsection{Sectoral volumes}
Most slides do not involve the entire seamount, of course. Instead, they only affect a limited sector of it.  Thus, our slide sector may be specified by two
azimuths $\alpha_1$ and $\alpha_2$ and the slide will only affect that part of the seamount with a volume fraction
\begin{equation}
\theta = \frac{\alpha_2 - \alpha_1}{360}.
\end{equation}
In reporting volumes we therefore need to scale $V_s$ by $\theta$, but the equations above for balancing volumes are not affected since $\theta$ cancels from both sides of the equations.
However, $\theta$ will enter in the next section when a specific volume is requested.

\subsection{Specify the slide volume}

We may we want a slide that redeposits a certain fraction $\phi_0$ of the total volume $V_0$ of the entire seamount, or
a specific volume $V$. In that case we will need to compute $u_0$ given the other parameters in order to match the volumes.  Using equation (\ref{eq:Vs})
for $V_s$ we equate it to $\phi_0 V_0$ (or to a specified volume) and rearrange the equation into a form where the right-hand side is independent of $u_0$:
\begin{equation}
\left ( r_2 + \Delta r \bar{u}_s^q \right ) u_0 \left [ (1 + u_0) \log \left (\frac{1 + u_0}{u_0} \right ) - 1 \right ] = \frac{1}{2\Delta h} \left [\frac{V_f - V_s/\theta}{\pi \Delta r} - h_1(r_1 + r_2) \right ].
\label{eq:u0}
\end{equation}
Note we scale $V_s$ to a full 360-range since that is the limit used for the other volume fractions.  We then solve for $u_0$ numerically.
We note that $V_s/\theta$ cannot exceed $V_f$ so $\phi_0$ is limited by the other parameters.  As $V_s \rightarrow V_f$ we find $u_0 \rightarrow 0$.

\subsection{Evolution of a slide over time}

\begin{figure}[h!]
  \centering
  \includegraphics[width=5in]{slides_psi.png}
  \caption{We can control how quickly a slide happens by manipulating the $\psi(\tau)$ function. A
  linear curve means the mass redistribution is taking place at a constant rate during the slide duration.
  Adjust $\beta$ to have the bulk of the redistribution happen at the front ($\beta < 1$) or closer
  to the end ($\beta > 1$).}
  \label{slides_psi}
\end{figure}

The above results discuss the final shape of topography after a slide has occurred.  However, for time-series analysis
we will also need to look at intermediate stages.  This brings up a few further adjustments. We will assume that a
slide starts at $t_0$ and completes by $t_1$.  We will introduce $\psi(\tau) = \tau^\beta$ as the normalized volume fraction of the slide
as a function of normalized time $\tau = (t - t_0)/(t_1 - t_0)$.  Hence, $\psi(0) = 0$ and $\psi(1) = 1$ and
in between we require it to be a monotonically increasing curve (e.g., linear ($\beta = 1$); see Figure~\ref{slides_psi}).  This means that, for some time $t$, the actual
volume is the fraction $\psi(t)$ of the final slide volume $V_{s_0}$. Because our radial slide function $h_s(r)$ starts at the top scarp ($h_2$) and
ends at the bottom ($h_1$) we realize that to compute a slide with a reduced volume given by $\psi(t)$ we must
recompute the $u_0$ value initially assigned via (\ref{eq:u0}). There are two separate cases to consider:
\begin{enumerate}
  \item If a specific final landslide volume has been prescribed via $\phi_0$ then for time $t$ we simply use $\phi = \phi_0 \psi(t)$
  when computing $u_0$.
  \item If instead the final landslide volume is \emph{not} set but we provided initial parameters such as $u_0$, then we first need to determine
  what the final slide volume $V_{s_0}$ would be from (\ref{eq:Vs}), then assign the equivalent volume fraction  $\phi_0 = V_{s_0}/V_0$, and
  again use $\phi = \phi_0 \psi(t)$ to compute the changing $u_0$ parameter.
\end{enumerate}
At each time step we must therefore recompute the distal radius $r_d$ but keep the start of the toe deposit at $h_c$.

\subsection{Azimuthal variation in slide shape}

\begin{figure}[h!]
  \centering
  \includegraphics[width=5in]{slides_azim.png}
  \caption{A range of azimuthal variation in slide height can be achieved by modulating the power parameter, $p$. This variation
  means the slide volume is reduced by $1 - \bar{s}$ (dashed lines). E.g., for $p = 2$ the slide volume is only 67\% of the
  volume we would have if there was no azimuthal variation (i.e., $s = 1$).}
  \label{slides_azim}
\end{figure}

The above derivations assume the slide profile $h_s(r)$ only varies with radial position and is constant as a function of the azimuth within the slide sector.
Alternatively, we may wish to have the slide start at azimuth $\alpha_1$ at the undeformed scarp height ($h(r)$) and drop down to the final level ($h_s(r)$)
over some azimuthal distance towards $\alpha_2$.  We model this azimuthal scaling function by
\begin{equation}
s(\alpha) = s(\gamma) = 1 - \left |\gamma\right|^p,
\end{equation}
which, for $p = 2$, yields a concave parabolic function. The normalized angle $\gamma$ over the range $\pm1$ is defined as
\begin{equation}
\gamma = 2\frac{\alpha - \alpha_1}{\alpha_2 - \alpha_1} - 1.
\end{equation}
We can use the power modulator $p \ge 2$ to increase the drop-off rate from the scarp toward the middle of the section. Now, the computed slide height will be a function
of both radius and azimuth, i.e.,
\begin{equation}
h_s(r, \alpha) = h(r) \left [1 - s(\alpha)\right ] + h_s(r) s(\alpha),
\end{equation}
When $p \rightarrow \infty$ then $s(\alpha) \rightarrow 1$ and we recover the original radial-variation-only slide height. For other values,
see Figure~\ref{slides_azim}.
For volume calculations we must integrate over the azimuthal range, which yields
\begin{equation}
\bar{s} = \int_0^1  \gamma^p d\gamma = \frac{p}{p+1},
\end{equation}
since $p \ge 2$. Now, $\bar{s}$ can be used to evaluate the actual slide volume, i.e., we must adjust the previously obtained slide volume:
\begin{equation}
V_s^' = \bar{s} V_s = \bar{s} \left (V_f - V^u_q - V^l_q \right).
\end{equation}
On the other hand, if specific slide volume fractions are requested then we need to scale $\phi_0$ by $1/\bar{s}$ so that the
volume reduction due to $s(\alpha)$ is exactly counter-balanced by the increase in requested volume.

\section{Volumes given specific seamount shapes}

In order to use Pappas' theorem for computing the flank volume $V_f = 2 \pi \bar{r}_f A_f$ we need analytical
solutions for the various areas and centroid distances, which depend on the seamount shape model.  The
superscripts below refer to the shape used (c, p, g, or o). Here,
we redefine $u = r/r_0$ so $r = r_0 u$ and $dr = r_0 du$.  Integrations still go from $u_2$
to $u_1$.  We let $r_0 = h_0 = 1$ and then $\bar{r}_f = r_0 \bar{u}_f$ and we must scale the nondimensional areas $K$
by $r_0$ and the relevant height factor (which differs for each shape) to get actual areas. We note that
in all cases, the slide escarpment will start beyond the truncated surface, i.e., $r_2 \ge r_f$, hence we avoid
any complications with $h(r$) being horizontal within the range of the slide.

\subsection{Conic Seamounts (c)}

For conic seamounts we use normalized height $v(u) = 1 - u$. Then
\begin{equation}
K^c = \int_{u_2}^{u_1} v(u) du = u_1 - u_2 - \frac{1}{2}\left ( u_1^2 - u_2^2 \right ), \quad \bar{u}_f^c = \frac{\int_{u_2}^{u_1} v(u) u du}{K^c} = \frac{3(u_1^2 - u_2^2) - 2 (u_1^3 - u_2^3)}{6K^c},
\end{equation}
\begin{equation}
A_f^c = \frac{h_0 r_0}{1-f}K^c, \quad r_f^c = r_0\bar{u}_f^c.
\end{equation}

\subsection{Parabolic Seamounts (p)}

For parabolic seamounts we use normalized height $v(u) = 1 - u^2$. Then
\begin{equation}
K^p = \int_{u_2}^{u_1} v(u) du = u_1 - u_2 - \frac{1}{3}\left ( u_1^3 - u_2^3 \right ), \quad \bar{u}_f^p = \frac{\int_{u_2}^{u_1} v(u) u du}{K^p} = \frac{2(u_1^2 - u_2^2) - (u_1^4 - u_2^4)}{4K^p},
\end{equation}
\begin{equation}
A_f^p = \frac{h_0 r_0}{1-f^2}K^p, \quad r_f^p = r_0\bar{u}_f^p.
\end{equation}

\subsection{Gaussian Seamounts (g)}

For Gaussian seamounts we use normalized height $v(u) = e^{-\frac{9}{2}u^2}$. Then

\begin{equation}
K^g = \int_{u_2}^{u_1} v(u) du = \frac{\sqrt{2\pi}}{6} \left [ \mbox{erf} \left (\frac{3\sqrt{2}}{2}u_1\right ) - \mbox{erf} \left (\frac{3\sqrt{2}}{2}u_2\right ) \right ],
\end{equation}
\begin{equation}
\bar{u}_f^g = \frac{\int_{u_2}^{u_1} v(u) u du}{K^g} = \frac{e^{-\frac{9}{2}u_2^2} - e^{-\frac{9}{2}u_1^2}}{9K^g},
\end{equation}
\begin{equation}
A_f^g = h_0 r_0 e^{\frac{9}{2}f^2} K^g, \quad r_f^g = r_0\bar{u}_f^g.
\end{equation}

\subsection{Polynomial Seamounts (o)}

For polynomial seamounts we use the normalized height
\begin{equation}
v(u) = \frac{(1 + u)^3 (1 - u)^3}{1 + u^3}.
\end{equation}
Then,
\begin{equation}
K^o = \int_{u_2}^{u_1} v(u) du = u_1 - u_2 + \frac{3}{2}\left (u_1^2 - u_2^2 \right ) - \frac{1}{4} \left (u_1^4 - u_2^4\right ) - L - T,
\end{equation}
\begin{equation}
\bar{u}_f^o = \frac{\int_{u_2}^{u_1} v(u) u du}{K^o} = \frac{- 3 (u_1 - u_2) + \frac{1}{2}(u_1^2 - u_2^2) + (u_1^3 - u_2^3) - \frac{1}{5}(u_1^5 - u_2^5) - L + T}{K^o},
\end{equation}
where 
\begin{equation}
L = \frac{3}{2} \log \left ( \frac{u_1^2 - u_1 + 1}{u_2^2 - u_2 + 1}\right ), \quad T = \sqrt{3} \left [ \tan^{-1} \left (\frac{\sqrt{3}}{3}(2u_1 - 1)\right ) - \tan^{-1} \left (\frac{\sqrt{3}}{3}(2u_2 - 1)\right )\right ].
\end{equation}
Hence,
\begin{equation}
A_f^o = \frac{h_0 r_0}{v(f)} K^o, \quad r_f^o = r_0\bar{u}_f^o.
\end{equation}

\subsection{Special corner cases}

The radial slide function $q(u)$ can, for $u_0 \rightarrow \infty$ get close to a straight line and thus is a perfect
initial starting shape for a slide off a conic seamount.  However, other seamount shapes are not linear over the range where the slide
occurs, and hence there is a specific amount of volume between the limiting straight $q(u)$ curve and the undeformed seamount
flank $h(r)$; we call this volume $V_m^w$ and it is a function of the seamount shape $w$.  For the cone it is identically zero.
In general, the surplus or deficit volume is
\begin{equation}
V_m^w = V_f^w - V_f^c,
\end{equation}
where seamount shape $w$ is one of (c, p, g, or o).  By examining if the shapes of these seamounts are convex or concave
we see that $V_m^p > 0$, $V_m^c = 0$, and both $V_m^g < 0$ and $V_m^o < 0$. This does not matter in the case where we
specify the parameter $u_0$ since we do not place any limit on the volume itself (there may be, however, a limit on
how large a $u_0$ we can prescribe). Instead, these volume differences are tied to cases where a specific slide volume is requested:

\begin{itemize}
  \item For the parabolic shape with a volume surplus, even a value of $u_0 = \infty$ (no slide yet) will still
  produce a slide volume $V_s$ since the curve $q(u$) is below the flank.  This means we cannot model very small-volume
  slides beneath the $V_m^p$ limit using our $q(u)$ model. In effect, the slide volume starts at $V_m^p$ and grows from there.
  Consequently, there is a finite lower limit on $\phi$ that we must exceed.  We will develop a fix below.
  \item For the two shapes with volume deficits, initially large values of $u_0$ will not actually result in any slide
  volume $V_s$ since the curve $q(u$) is still outside the flank. This means $u_0$ has to increase to a certain level
  before we get a positive slide volume. Consequently, there is a finite upper limit on $u_0$ value that we cannot exceed.
\end{itemize}

For time-series simulation we compute volume fractions via $\psi(\tau)$, and in the first case those fractions may
be too small for early time-steps to allow us to solve for $u_0$. For these parabolic seamounts with slide fractions
that cannot be modeled via $u_0$, we can do this: Let $\eta = V_s(t)/V_m^p < 1$ be the fraction of the volume and use it to move
$h_s(r)$ \emph{outward} towards $h(r)$:
\begin{equation}
h_s(r) = h_c(r) + \eta \left [h(r) - h_c(r) \right ],
\end{equation}
where $h_c(r)$ is the straight line between $(r_1, h_1)$ and $(r_2, h_2)$.  This will let us effectively carve out
those smaller volumes that otherwise cannot be done via $u_0$.  The second case is not really a problem since there is
always a $u_0$ we can solve for.

\section{REFERENCES}

Smith, J. R., and P. Wessel (2000), Isostatic consequences of giant landslides on the Hawaiian Ridge,
{\it Pure Appl. Geophys., 157}, 1097--1114, doi:10.1007/s000240050019.
\end{document}
