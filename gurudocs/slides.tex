%	$Id$
% Internal documentation for slide modeling in grdseamount
% Paul Wessel, November 16, 2016.
\documentclass[12pt,letterpaper,margin=0.5in]{report}
\usepackage{times}
\usepackage{graphicx}
\usepackage{breqn}
\usepackage[margin=0.5in]{geometry}
\usepackage{lscape}
\textheight = 9 in
\topmargin = -1 in
\begin{document}

\section{Model curve for a landslide in grdseamount}

\begin{figure}[h!]
  \centering
  \includegraphics[width=5in]{slides_fig1.png}
  \caption{Geometry for an \emph{ad hoc} landslide approximation in \texttt{grdseamount}.  The slide material (red hachured area)
  will be deposited at the toe of the seamount (blue hachured area) reaching up to a height of $h_c$ and linearly
  tapering to zero at a distal point $r_d$. Note that $h_2 > h_1$ while $r_1 > r_2$.}
  \label{slides_fig1}
\end{figure}

These notes resurrect the work of {\it J. Smith and Wessel} (2000) on approximating large landslides off seamounts
for the purpose of studying the isostatic consequences.  I found the old C codes and wanted to document the geometry
better as well as implement it as a new option in grdseamount.  I refer to Figure~\ref{slides_fig1}.  The equation for the
shape of the slide for the radial range $r_2$ to $r_1$ is given by a single equation:
\begin{equation}
h_s(r) = h_1 + \Delta h \cdot q(u),
\end{equation}
where $\Delta h = h_2 - h_1$ is the \emph{height} range of the slide and $q(u)$ is the normalized shape curve of the slide:
\begin{equation}
q(u) = u_0 \left (\frac{1 + u_0}{u + u_0} - 1 \right ).
\end{equation}
This is basically a hyperbolic curve and we can use the tuning parameter $u_0$ to carve more or less into the seamount core,
which is an improvement upon the fixed scheme in {\it J. Smith and Wessel} (2000).
Here, $u$ is the normalized distance from $r_2$ to $r_1$, and over this range $q$ goes from 1 to 0:
\begin{equation}
u = \frac{r-r_2}{\Delta r},
\end{equation}
where $\Delta r = r_1 - r_2$ is the \emph{radial} range of the slide.
Figure~\ref{slides_fig2} shows the typical shapes of $q(u)$ for a range of $u_0$ values.
\begin{figure}[h!]
  \centering
  \includegraphics[width=5in]{slides_fig2.png}
  \caption{A variety of slide shapes are possible by varying $u_0$.  The slide area for a conical seamount would be the area
  between the flank (dashed line) and the selected curve.}
  \label{slides_fig2}
\end{figure}

Depending on the shape of the seamount prior to the slide event (called $h(r)$, with $h_s(r) \le h(r)$), we will need to compute the various radii referred
to in Figure \ref{slides_fig1}, such as $r_1$, $r_2$, $r_c$, and $r_d$ from their corresponding heights $h_1$, $h_2$, $h_c$ and volume of slide.
The material removed by the land slide (red hachured area)
will be deposited at the base of the seamount (blue hachured area) up to the proximal height $h_c$.  We assume this deposit will have a linearly
decaying height, enabling us to compute the distal radius $r_d$ from the volume of the slide, $V_s$.  Clearly $V_s$ depends
on both $h(r)$ and $h_s(r)$.  We will compute this volume as $V_s = V_f - V_q$, where $V_f$ is the flank volume whose
triangular (for a cone) crossection is delimited by the lines $r = r_2$, $h = h_1$ and $h(r)$.  Its area $A_f$ and centroid $\bar{r}_f$ are computed once and
we then use Pappas' theorem to get the corresponding volume $V_f = 2 \pi \bar{r}_f A_f$.  For $V_s$ the upper limit is $h_s(r)$ instead of $h(r)$ so we can integrate
$h_s(r)$ for an analytical answer. First we get the area:
\begin{equation}
A_s = \int_{r_2}^{r_1} h_s(r) dr = \Delta h \Delta r \int_0^1 q(u) du = \Delta h \Delta r u_0 \left [ (1 + u_0) \log \left (\frac{1 + u_0}{u_0} \right ) - 1 \right ].
\end{equation}
To use Pappas' theorem for computing the volume we need the radius to the centroid, which is obtained via
\begin{equation}
\bar{r}_s = \frac{\int_{r_2}^{r_1}h_s(r)rdr}{\int_{r_2}^{r_1}h_s(r)dr} = r_2 + \Delta r \bar{u}_s = r_2 + \Delta r \frac{\int_0^1q(u)udu}{\int_0^1 q(u)du}.
\end{equation}
This integral yields
\begin{equation}
\bar{u}_s = \frac{2(1 + u_0)\left [1 - u_0 \log \left ( \frac{1+u_0}{u_0} \right ) \right ] - 1}{(1 + u_0) \log \left (\frac{1 + u_0}{u_0} \right ) - 1}.
\end{equation}
Our final slide volume is therefore
\begin{equation}
V_s = V_f - 2 \pi \Delta h \Delta r \left ( r_2 + \Delta r\bar{u}_s \right ) u_0 \left [ (1 + u_0) \log \left (\frac{1 + u_0}{u_0} \right ) - 1 \right ].
\end{equation}
To solve for the distal end radius $r_d$ we need to equate the volume $V_s$ with the equivalent distal volume.
Given that the distal triangle area\footnote{We assume that $h(r)$ from $h_c$ to zero may be approximated as a linear ramp.} is
\begin{equation}
A_d = \frac{h_c}{2} (r_d - r_0),
\end{equation}
we find the volume to be
\begin{equation}
V_d = 2 \pi \bar{r}_d A_d = 2 \pi \bar{r}_d \left [ \frac{h_c}{2} (r_d - r_0) \right ],
\end{equation}
where $\bar{r}_d$ is the centroid radial distance of the distal triangle and is obtained in a similar fashion to $\bar{u}$ (except
we must handle the awkward initial section separately):
\begin{equation}
\bar{r}_d = \frac{\int_{r_c}^{r_d}h_c \left (1 - \frac{r - r_c}{r_d - r_c} \right )rdr - \int_{r_c}^{r_o}h_c \left (1 - \frac{r - r_c}{r_o- r_c} \right )rdr}{A_d}
\end{equation}
Simplifying this expression and equating the two volumes lead to the quadratic equation
\begin{equation}
r_d^2 + r_c r_d - \left (r_0^2 + r_0 r_c + \frac{3 V_s}{\pi h_c}\right ) = r_d^2 + r_c r_d - c = 0,
\end{equation}
with unique solution
\begin{equation}
r_d = \frac{-r_c + \sqrt{r_c^2 + 4c}}{2}.
\end{equation}

We may we want a slide that redeposits a certain fraction $\phi$ of the total volume $V_0$ of the entire seamount. In that
case we will need to compute $u_0$ given the other parameters in order to match the volumes.  We have from above
\begin{equation}
V_s = \phi V_0 = V_f - 2 \pi \Delta h \Delta r \left ( r_2 + \Delta r\bar{u}_s \right ) u_0 \left [ (1 + u_0) \log \left (\frac{1 + u_0}{u_0} \right ) - 1 \right ].
\end{equation}
We rearrange this equation into the form
\begin{equation}
\left ( r_2 + \Delta r \bar{u}_s \right ) u_0 \left [ (1 + u_0) \log \left (\frac{1 + u_0}{u_0} \right ) - 1 \right ] = \frac{V_f - \phi V_0}{2 \pi \Delta h \Delta r}
\end{equation}
and solve for $u_0$ numerically.  We note that $V_s$ cannot exceed $V_f$ so $\phi$ is limited by the other parameters.  As
$V_s$ approaches $V_f$ we will get diminishing values for $u_0$.

Most slides do not involve the entire seamount, but instead only affects a limited sector of it.  Thus, our slide sector may be specified by two
azimuths $\alpha_1$ and $\alpha_2$ and the slide will only affect that part of the seamount with a volume fraction
\begin{equation}
\theta = \frac{\alpha_2 - \alpha_1}{360}.
\end{equation}
In reporting volumes we need to scale $V_s$ by $\theta$ but the equations above for balancing volumes are not affected since $\theta$ would cancel from both sides of the equations.

\section{Volumes given specific seamount shapes}

In order to use Pappas' theorem to compute the flank volume $V_f = 2 \pi \bar{r}_f A_f$ we need analytical
solutions for the various areas and centroid distances.  These are determined in this section. The superscripts
on the left-hand side in these sections refer to the seamount shape used (c, p, g, o).

\subsection{Conic Seamounts}

For conic seamounts we find
\begin{equation}
A_f^c = \frac{h_0 \Delta_r^2}{2 r_0 (1-f)}, \quad \bar{r}_f^c = \frac{2 r_2 + r_1}{12}.
\end{equation}

\subsection{Parabolic Seamounts}

For parabolic seamounts we find
\begin{equation}
A_f^p = \frac{h_0 \left ( r_1^3 - 3 r_1^2 r_2 + 2 r_2^3\right)}{6 r_0^2 (1-f)}, \quad \bar{r}_f^p = \frac{\left (r_1^2 - r_2^2 \right )^2}{24 \left ( r_1^3 - 3 r_1^2 r_2 + 2 r_2^3\right)}.
\end{equation}

\subsection{Gaussian Seamounts}

For Gaussian seamounts we find
\begin{equation}
A_f^g = h_0 r_0 e^{\frac{9}{2}f^2} \left \{ \frac{\sqrt{2\pi}}{6} \left [\mbox{erf} \left ( \frac{3\sqrt{2}}{2}u_2\right ) - \mbox{erf} \left ( \frac{3\sqrt{2}}{2}u_1 \right ) \right ] - (u_1 - u_2)e^{-\frac{9}{2} u_1^2} \right \}.
\end{equation}
\begin{equation}
\bar{r}_f^g = r_0\frac{\frac{1}{9} \left [e^{-\frac{9}{2}u_2^2} - e^{-\frac{9}{2}u_1^2} \right ] - \frac{1}{2}\left (u_1^2 - u_2^2 \right )e^{-\frac{9}{2}u_1^2}}{\frac{\sqrt{2\pi}}{6}\left [\mbox{erf} \left ( \frac{3\sqrt{2}}{2}u_2\right ) - \mbox{erf} \left ( \frac{3\sqrt{2}}{2}u_1 \right ) \right ] - (u_1 - u_2)e^{-\frac{9}{2}u_1^2}}.
\end{equation}

\subsection{Polynomial Seamounts}

For polynomial seamounts we set the stage with the function for the shape function
and two needed integrals which we solved in Matlab. The untruncated flank shape is
\begin{equation}
p(u) = \frac{(1 + u)^3 (1 - u)^3}{1 + u^3},
\end{equation}
while the integral related to area is
\begin{equation}
\int \left ( p(u) - p(u_1)\right ) du = u \left(1 - p(u_1)\right) + \frac{3}{2}u^2 - \frac{1}{4}u^4 - \frac{3}{2} \log (u^2 - u + 1) - \sqrt{3} \tan^{-1} \left ( \frac{\sqrt{3}}{3}(2u - 1) \right),
\end{equation}
and the integral related to the centroid is
\begin{equation}
\int \left ( p(u) - p(u_1)\right ) u du =  -3 u + \frac{u^2}{2}\left(1 - p(u_1)\right) +u^3 - \frac{u^5}{5} - \frac{3}{2} \log (u^2 - u + 1) + \sqrt{3}\tan^{-1} \left ( \frac{\sqrt{3}}{3}(2u - 1) \right).
\end{equation}
Introducing $u_1 = r_1/r_0$ and $u_2 = r_2 / r_0$ we find
\begin{equation}
A_f^o = \frac{h_0 r_0}{p(f)} \left \{ (u_1 - u_2)(1 - p(u_1)) + \frac{3}{2}\left (u_1^2 - u_2^2\right ) - \frac{1}{4}\left (u_1^4 - u_2^4\right ) -  L - T \right \}
\end{equation}
where 
\begin{equation}
L = \frac{3}{2} \log \left ( \frac{u_1^2 - u_1 + 1}{u_2^2 - u_2 + 1}\right ), \quad T = \sqrt{3} \left [ \tan^{-1} \left (\frac{\sqrt{3}}{3}(2u_1 - 1)\right ) - \tan^{-1} \left (\frac{\sqrt{3}}{3}(2u_2 - 1)\right )\right ].
\end{equation}
For the centroid location we first note
\begin{equation}
\bar{r}_f^o = r_2 + \Delta r \bar{u}_f^o = r_2 + \Delta r \frac{\int_{u_2}^{u_1} \left ( p(u) - p(u_1)\right ) u du}{\int_{u_2}^{u_1} \left ( p(u) - p(u_1)\right )du},
\end{equation}
which yields
\begin{equation}
\bar{u}_f^o = \frac{ - 3 (u_1 - u_2) + \frac{1 - p(u_1)}{2}(u_1^2 - u_2^2) + (u_1^3 - u_2^3) - \frac{1}{5}(u_1^5 - u_2^5) - L + T}{(u_1 - u_2)(1 - p(u_1)) + \frac{3}{2}\left (u_1^2 - u_2^2\right) - \frac{1}{4}\left(u_1^4 - u_2^4\right ) - L - T}.
\end{equation}

\section{Azimuthal variation}

\begin{figure}[h!]
  \centering
  \includegraphics[width=5in]{slides_fig3.png}
  \caption{A range of azimuthal variation in slide height can be achieved via the modulating power parameter $v$.}
  \label{slides_fig3}
\end{figure}

The above derivations assume the slide profile $h_s(r)$ only varies with radial position and is constant as a function of the azimuth within the slide sector.
Alternatively, we may wish to have the slide start at azimuth $\alpha_1$ at the undeformed scarp height ($h(r)$) and drop down to the final level ($h_s(r)$)
over some azimuthal distance towards $\alpha_2$.  We model this azimuthal scaling function by
\begin{equation}
s(\alpha) = s(\gamma) = \left |\gamma\right|^v,
\end{equation}
which for $v = 2$ yields a parabolic function, and the normalized angle $\gamma$ on the range $\pm1$ is defined as
\begin{equation}
\gamma = 2\frac{\alpha - \alpha_1}{\alpha_2 - \alpha_1} - 1.
\end{equation}
We can use the power modulator $v \ge 2$ to increase the drop-off rate from the scarp toward the middle of the section. Now, the computed slide height will be a function
of both radius and azimuth, i.e.,
\begin{equation}
h_s(r, \alpha) = h(r) s(\alpha) + h_s(r) (1 - s(\alpha)),
\end{equation}
When $v \rightarrow \infty$ then $s(\alpha) \rightarrow 0$ and we recover the original radial-variation-only slide height. For other values,
see Figure~\ref{slides_fig3}.
For volume calculations we must integrate over the azimuthal range, which yields
\begin{equation}
\bar{s} = 2\int_0^1  \gamma^v d\gamma = \frac{2}{v+1},
\end{equation}
since $v \ge 2$. Now, $1 - \bar{s}$ can be used to evaluate the actual slide volume, i.e., we must scale the previously obtained volume:
\begin{equation}
V_s^' = V_f - (1 - \bar{s}) V_s.
\end{equation}

\section{REFERENCES}

Smith, J. R., and P. Wessel (2000), Isostatic consequences of giant landslides on the Hawaiian Ridge,
{\it Pure Appl. Geophys., 157}, 1097--1114, doi:10.1007/s000240050019.
\end{document}
