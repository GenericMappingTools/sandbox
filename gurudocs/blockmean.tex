%	$Id$
% Internal documentation for blockmean
% Paul Wessel, May 23, 2018.
\documentclass[12pt,letterpaper,margin=0.5in]{report}
\usepackage{times}
\usepackage{graphicx}
\usepackage{breqn}
\usepackage[margin=0.5in]{geometry}
\usepackage{lscape}
\textheight = 9 in
\topmargin = -1 in
\begin{document}

\section*{BLOCKMEAN}

Way back in 1987 or so, Walter built the blockm* modules as pre-processors to gridding routines as a way to avoid spatial aliasing
and possibly remove outliers.  However, these modules can also be considered standalone tools for various
statistical operations on points within spatial bins.  This dual use has lead to some issues for blockmean that need to
be discussed and the outputs need better documentation.  We may also need to provide new options or modifiers
to allow the full range of possible outputs while retaining backwards compatibility.

\subsection*{1. Weights}
As originally implemented, blockmean expected $(x_i,y_i,z_i)$ triplets and returned ($\bar{x}, \bar{y}, \bar{z}$)
per bin containing data.
However, if the {\bf --W} option is used then we expect $(x_i,y_i,z_i,w_i)$ records instead and compute the weighted means
using the relation
\begin{equation}
	\bar{\phi} = \frac{\sum_i^n \phi_i w_i}{\sum_i^n w_i} = \frac{\sum_i^n \phi_i w_i}{S_w},
	\label{eq:weighted}
\end{equation}
where $\phi$ is the variable we want to average.  That is, the weights are applied when computing any of  ($\bar{x}, \bar{y}, \bar{z}$).
We also output the sum of the weights, $S_w$, unless it is suppressed by using {\bf--Wi}.  The idea
behind outputting the \emph{sum of the weights} was to facilitate grand means at a later stage by combining several
intermediate results via a second run of blockmean.  Here, the weight sums would be used as input weights in the second
run and hence the sum rather than a mean weight is required.
Weights or not, ($\bar{x}, \bar{y}$) can be replaced by the coordinates of the bin center if {\bf --C} is set.

\subsection*{2. Uncertainties}

At some point we started to allow input of the form  $(x_i,y_i,z_i,\sigma_i)$ via the {\bf +s} modifier to {\bf--W}. Now,
the given data uncertainties are converted to weights via $w_i = 1/\sigma_i$ and weighted means are computed
as per (\ref{eq:weighted}).  However, the least squares solution for the weighted mean actually requires $w_i = 1/\sigma_i^2$
({\bf ISSUE 1}). Since $\sigma_i$ has the same units as $z_i$ it no longer makes sense to use these weights
for computing ($\bar{x}, \bar{y}$), but we do not explicitly prevent it; I think we should ({\bf ISSUE 2}).
Finally, the output weights are now reported as $1/S_w$.  It is not clear how useful this is ({\bf ISSUE 3}).
Given those weights, the weighted mean for the data values should be computed via (\ref{eq:weighted}) as
\begin{equation}
	\bar{z} = \frac{\sum_i^n z_i \sigma_i^{-2}}{\sum_i^n \sigma_i^{-2}} = \sum_i^n z_i \alpha_i,
	\label{eq:w}
\end{equation}
where
\begin{equation}
	\alpha_i = \frac{ \sigma_i^{-2}}{\sum_i^n \sigma_i^{-2}} = \frac{1}{\sigma_i^2 S_w}.
	\label{eq:a}
\end{equation}


\subsection*{3. Extended Output}

We also added an option {\bf --E} for \emph{extended output}.  In addition to the mean triplet reported above, we append
the values $s, l, h$, where $s$ is the [weighted] sample standard deviation of the points inside a bin, and $l$ and
$h$ are the lowest and highest values encountered in that bin.

\subsection*{4. Error Propagation}

In 2013, John Robbins (Feature request \#217) wanted us to extend blockmean to do error propagation on the mean.  We implemented
this via {\bf --Ep}, which implied three additional conditions:
\begin{itemize}
	\item It requires the input weight option ({\bf --W} or {\bf --Wi}) to be selected.
	\item Awkwardly, it requires the input uncertainties $\sigma_i$ to be passed as weights $w_i$ (but externally computed as $\sigma_i^{-2}$).
		This seems an odd choice instead of letting blockmean accept $\sigma_i$ (as above) and do that conversion internally via {\bf --W+s} ({\bf ISSUE 4}).
	\item The output $s$ value is now the propagated uncertainty on the mean.  However, the current implementation (following Robbins)
		uses the equation $\sqrt{\sum \sigma_i^2}/n$, which is the equation for the simple mean but not the \emph{weighted} mean ({\bf ISSUE 5}).
\end{itemize}
We note there is no documentation if both {\bf --Ep} and {\bf --W+s} are given and what that would mean for the calculations ({\bf ISSUE 6}).

Given (\ref{eq:w}) and (\ref{eq:a}) we assume independent errors and obtain the uncertainty in the weighted $\bar{z}$ due to the individual uncertainties as
\begin{equation}
	\sigma_{\bar{z}}^2 = \sum_i^n \left (\alpha_i \sigma_i \right )^2 = \sum_i^n \left (\frac{1}{\sigma_i^2 S_w} \sigma_i \right )^2 = \sum_i^n \frac{1}{\sigma_i^2 S_w^2} = \frac{1}{S_w}.
	\label{eq:v}
\end{equation}
Hence, the uncertainty in the weighted mean should instead be reported as
\begin{equation}
	\sigma_{\bar{z}} = \frac{1}{\sqrt{S_w}}.
	\label{eq:s}
\end{equation}

\subsection*{5. Proposed Modifications}

We propose to address the six issues thus:
\begin{enumerate}
	\item When data uncertainties are given ({\bf --W+s}), compute weights as $w_i = 1/\sigma_i^2$.
	\item Only compute weighted ($\bar{x}, \bar{y}$) when weights are given, and simple means when data uncertainties are given.
	\item Always output the \emph{weight sum} if output weights are requested.
	\item This issue is technically not a bug, but it is unusual since it clearly involves data uncertainties but expects to be
		given weights computed according to a specific recipe ($w_i = 1/\sigma_i^2$).  To remain fully backwards compatible I
		suggest we introduce {\bf --E+p} to capture the new behavior and stop document (but keep the code for) the old {\bf --Ep}.
		New {\bf --E+p} should handle weights [{\bf --W}] or uncertainties (via {\bf --W+s}).
	\item For error propagation ({\bf --Ep}), use (\ref{eq:s}) for the propagated uncertainty on weighted means.
	\item This becomes a compatibility issue under the hood for {\bf --Ep}.  Basically, {\bf --Ep} requires {\bf --W} and should give
		an error (with explanation) if given {\bf --W+s}.
\end{enumerate}
\end{document}
