%	$Id$
% Internal documentation for blockmean
% Paul Wessel, May 23, 2018.
\documentclass[12pt,letterpaper,margin=0.5in]{report}
\usepackage{times}
\usepackage{graphicx}
\usepackage{breqn}
\usepackage[margin=0.5in]{geometry}
\usepackage{lscape}
\textheight = 9 in
\topmargin = -1 in
\begin{document}

\section*{BLOCKMEAN}

Way back in 1987 or so, we built the blockm* modules as pre-processors to gridding routines as a way to avoid spatial aliasing
and possibly remove outliners.  However, these modules can also be considered standalone tools for various
statistical operations on points within spatial bins.  This dual use has lead to some issues that need to
be discussed and the result need better documentation.  We may also need to provide new options or modifiers
to allow the full range of possible outputs while retaining backwards compatibility.

\subsection*{Weights}
As originally implemented, blockmean expected $(x_i,y_i,z_i)$ triplets and returned ($\bar{x}, \bar{y}, \bar{z}$)
per bin containing data.
However, if the {\bf --W} option is used then we expect $(x_i,y_i,z_i,w_i)$ records instead and compute the weighted means
using the relation
\begin{equation}
	\bar{\phi} = \frac{\sum_i^n \phi_i w_i}{\sum_i^n w_i} = \frac{\sum_i^n \phi_i w_i}{S_w},
	\label{eq:weighted}
\end{equation}
where $\phi$ is the variable we want to average.  That is, the weights are applied when computing any of  ($\bar{x}, \bar{y}, \bar{z}$)
provided {\bf --C} is not used to center the output location at the middle of the bin.
In the latter case we also output the sum of the weights, $S_w$, unless it is suppressed by using {\bf--Wi}.  The idea
behind outputting the \emph{sum of the weights} was to facilitate grand means at a later stage by combining several
intermediate results via a second run of blockmean.  Here, the weight sums would be used as input weights in the second
run and hence the sum rather than a mean weight is required.

\subsection*{Uncertainties}

At some point we started to allow input of the form  $(x_i,y_i,z_i,\sigma_i)$ via the {\bf +s} modifier to {\bf--W}. Now,
the given data uncertainties are converted to weights via $w_i = 1/\sigma_i$ and weighted means are computed
as per (\ref{eq:weighted}).  Since $\sigma_i$ has the same units as $z_i$ it no longer makes sense to use these weights
for computing ($\bar{x}, \bar{y}$).  Futhermore, the output weights are now reported as $1/S_w$, which is possibly a bug.
Given these weights, the weighted mean for the data values is computed as
\begin{equation}
	\bar{z} = \frac{\sum_i^n z_i \sigma_i^{-1}}{\sum_i^n \sigma_i^{-1}} = \sum_i^n z_i \alpha_i,
	\label{eq:w}
\end{equation}
where
\begin{equation}
	\alpha_i = \frac{ \sigma_i^{-1}}{\sum_i^n \sigma_i^{-1}} = \frac{1}{\sigma_i S_w}.
	\label{eq:a}
\end{equation}


\subsection*{Extended output}

We also added an option {\bf --E} for extended output.  In addition to the mean triplet reported above, we append
the values $s, l, h$, where $s$ is the (weighted) standard deviation of the points inside a bin, and $l$ and
$h$ are the lowest and highest values encountered in that bin.

\subsection*{Error propagation}

In 2013, John Robbins (Feature request \#217) wanted us to extend blockmean to do error propagation on the mean.  We implemented
this via {\bf --Ep} which implied three things:
\begin{itemize}
	\item It requires that the input weight option ({\bf --W} or {\bf --Wi}) be selected.
	\item It requires the input uncertainties $\sigma_i$ to be passed as weights $w_i$ externally computed as $\sigma_i^{-2}$.
		This seems an odd choice instead of having blockmean accept $\sigma_i$ and do that conversion internally.
	\item The output $s$ value is now the standard deviation of the propagated error on the mean.  However, the
		implementation uses the equation $\sqrt{\sum \sigma_i^2}/n$, which seems wrong?
\end{itemize}
We note there is no check if {\bf --Ep} and {\bf --W+s} is given and what that would mean for the calculations.

Given (\ref{eq:w}) and (\ref{eq:a}) we assume independent errors and obtain the uncertainty in $\bar{z}$ due to the individual uncertainties:
\begin{equation}
	\sigma_{\bar{z}}^2 = \sum_i^n \left (\alpha_i \sigma_i \right )^2 = \sum_i^n \left (\frac{1}{\sigma_i S_w} \sigma_i \right )^2 = \sum_i^n S_w^{-2} = nS_w^{-2}.
	\label{eq:v}
\end{equation}
Hence the standard deviation in the weighted mean should instead be reported as
\begin{equation}
	\sigma_{\bar{z}} = \frac{\sqrt{n}}{S_w}.
	\label{eq:s}
\end{equation}

\subsection*{Proposed Modifications}


\section*{REFERENCES}

\end{document}
